\begin{eg}
\noindent \pmb{例:$Dirichlet$函数在$Riemann$意义下不可积.} 
\[ D(x) = 
\begin{cases} 
    1 & ,\text{if } x \in  [0,1] \cap \mathbb{Q} \\ 
    0 & ,\text{if } x \in [0,1]\setminus \mathbb{Q} 
\end{cases} \]
\end{eg}
\noindent 解:
\par
上积分:
\[\overline{\int_{a}^{b}} f(x) \,dx = \lim_{\left\lVert T\right\rVert  \to 0} \sum_{i = 1}^{n} M_{i}\Delta x_{i} = 1\]
\par
下积分:
\[\underline{\int_{a}^{b}} f(x) \,dx = \lim_{\left\lVert T\right\rVert  \to 0} \sum_{i = 1}^{n} m_{i}\Delta x_{i} = 0\]
\par
对 $\forall\left\lVert T\right\rVert$, 有
\[\sum_{i = 1}^{n} w_{i}\Delta x_{i} = 1\]
\par
第一充要条件和第二充要条件都不满足
\\ \hspace*{\fill}\\
\pmb{积分与极限交换次序(一般要求一致收敛)}
\begin{eg}
\pmb{例:设$\{x_{n}\}$为$[0,1]$中全体有理数,作$[0,1]$上的函数列}
\[ f_{n}(x) = 
\begin{cases} 
    1 & ,\text{if } x \in \{r_{1},r_{2},r_{3},\dots,r_{n}\}\\ 
    0 & ,\text{if } x \in [0,1]\setminus \{r_{1},r_{2},r_{3},\dots,r_{n}\}
\end{cases} \]
\end{eg}
\noindent 解:
\par
对$\forall x_{0} \in [0,1]$
\par
\ding{172} \ 若$x_{0} = r_{i} \ (i=1,\dots,n)$
\par
\quad \ 令$\delta = min\{\left\lvert x_{0}-r_{j}\right\rvert :j = 1,\dots,n,\land j\neq i\}$
\par
\quad \ 则$\forall x \in \cup _{\circ}(x_{0};\delta ) , f_{n}(x) = 0 ,\therefore \lim \limits_{x \to x_{0}}f_{n}(x)= 0$
\par
\quad \ $x_{0}$是$f_{n}(x)$的可去间断点
\par
\ding{173} \ 若$x_{0} \neq r_{i} \ (i=1,\dots,n)$
\par
\quad \ 令$\delta = min\{\left\lvert x_{0}-r_{i}\right\rvert :i = 1,\dots,n\} > 0$
\par
\quad \ 则$\forall x \in \cup(x_{0};\delta ) , f_{n}(x) = 0 ,\therefore \lim \limits_{x \to x_{0}}f_{n}(x)= 0 = f_{n}(x_{0})$
\par
\quad \ 故$f_{n}(x)$在$x_{0}$点处连续
\par
综上,$f_{n}(x)$在$[0,1]$有界且有且只有有限个间断点
\par
故$f_{n}(x)$在$[0,1]$上$Riemann$可积


\begin{eg}
\pmb{不$Riemann$可积:}
\[ \lim_{n \to \infty} f_{n}(x) = D(x) = 
\begin{cases} 
    1 & ,\text{if } x \in  [0,1] \cap \mathbb{Q} \\ 
    0 & ,\text{if } x \in [0,1]\setminus \mathbb{Q} 
\end{cases} \]
\end{eg}
\noindent 解:
\par
对$\forall x_{0} \in [0,1]$
\par
\ding{172} \ 若$x\notin \mathbb{Q} $
\par
\quad \ 则对$\forall n , f_{n}(x) = 0 , \ \therefore \lim \limits_{n \to \infty}f_{n}(x) = 0 = D(x)$
\par
\ding{173} \ 若$x\in \mathbb{Q} $
\par
\quad \ 则$\exists N \in \mathbb{N} ^{+} , s.t. \ x = r_{N}$
\par
\quad \ 对$\forall n > N , x \in \{r_{1},\dots,r_{n}\} , \ \therefore f_{n}(x) = 1$
\par
\quad \ 故$\lim \limits_{n \to \infty}f_{n}(x) = 1 = D(x)$
\\ \hspace*{\fill}\\
故对一般收敛函数列,在$Riemann$积分意义下极限运算与积分运算不一定可交换顺序
\\
即:
\[\lim_{n \to \infty} \int_{a}^{b} f_{n}(x) \,dx \stackrel{?}{=} \int_{a}^{b} \lim_{n \to \infty}  f_{n}(x) \,dx   \]

\newpage
\pmb{$Lebesgue$积分思想简介}
\begin{eg}
\pmb{问题:求$y = f(x)$,$x \in [a,b]$与$x$轴,$x = a$,$x = b$所围的曲边梯形面积}
\end{eg}
\noindent \pmb{法1:$Riemann$积分}
\par
作定义域$[a,b]$的分割$T$,$a = x_{0} < x_{1} < \dots <x_{n} = b$
\[M_{i} = \sup \limits_{[x_{i-1},x_{i}]}f(x)\]
\[m_{i} = \inf \limits_{[x_{i-1},x_{i}]}f(x)\]
\par
上和:\[\mathcal{S}(T) = \sum \limits_{i = 1}^{n}M_{i}\Delta x_{i}\]
\par
下和:\[S(T) = \sum \limits_{i = 1}^{n}M_{i}\Delta x_{i}\]
\par
若\[\lim \limits_{\left\lVert T\right\rVert  \to 0} \mathcal{S}(T) =  \lim \limits_{\left\lVert T\right\rVert  \to 0} S(T)\]
\par
则$f$在$[a,b]$上$Riemann$可积
\par
曲边梯形面积:\[S = (R) \int_{a}^{b} f(x) \,dx \]
\\
\pmb{法2:$Lebesgue$积分} 
\par
作值域$[c,d]$的分割$T$,$c = y_{0} < y_{1} < \dots <y_{n} = d$
\par
记
\[E_{i} = \{x \in [a,b] : y_{i-1} \leqslant f(x) \leqslant y_{i}\}\]
\par
积分和$\sum \limits_{i = 1}^{n} y_{i} \times E_{i} $的长度
\par
当$\left\lVert T \right\rVert \to 0$时,$S=\lim \limits_{\left\lVert T\right\rVert  \to 0} \sum \limits_{i = 1}^{n} y_{i} \times E_{i}$
\begin{eg}
\pmb{例:$Dirichlet$函数在$Lebesgue$意义下可积}
\[ D(x) = 
\begin{cases} 
    1 & ,\text{if } x \in  [0,1] \cap \mathbb{Q} \\ 
    0 & ,\text{if } x \in [0,1]\setminus \mathbb{Q} 
\end{cases} \]
\end{eg}
\noindent 解:
\par
对$\forall$值域分割$T$:有
\[0 = y_{0} < y_{1} < \dots <y_{n} = 1\]
\par
那么
\[E_{i} = \{x \in [a,b] : y_{i-1} \leqslant f(x) \leqslant y_{i}\}\]
\par
所以
\begin{center}
$\lim \limits_{\left\lVert T\right\rVert  \to 0} \sum \limits_{i = 1}^{n} y_{i} \times E_{i}$的长度$=$ $[0,1] \cap \mathbb{Q} $的长度
\end{center}
