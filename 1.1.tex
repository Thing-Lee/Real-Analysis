\pmb{本节重点:集合语言表示数学问题}
\begin{eg}
\pmb{例9:}若$f(x)$在$\mathbb{R}$上定义,且在$\mathbb{R}$上有上界$M$,则\[\mathbb{R} = \underbrace{\{x : f(x) \leqslant M\}}_\text{A} = \underbrace{\{x : f(x) \leqslant M + 1\}}_\text{B}\]
\end{eg}
\noindent 证明:
\par
显然:
\[A\subset B\subset \mathbb{R}\]
\par
下证:
\[\mathbb{R}\subset A\]
\par
因为$f(x)$在$\mathbb{R}$上有上界$M$
\par
故对$\forall x \in \mathbb{R}$,有$f(x) \leqslant M$
\par
故$\mathbb{R} \subset A$
\par
故$A = B = \mathbb{R}$
\begin{eg} 
\pmb{例10:}若$f(x)$在$\left[a,b\right]$上连续,则$f(\left[a,b\right]) = \left[m,M\right]$,其中$m = \min \{f(x) : x \in \left[a,b\right]\}$,$M = \max \{f(x) : x \in \left[a,b\right]\}$
\end{eg}
\noindent 证明:
\par \ding{172} 对$\forall x \in \left[a,b\right]$,有$m \leqslant f(x) \leqslant M$
\par \quad 所以$f(\left[a,b\right] \subset \left[m,M\right])$
\par \ding{173} 对$\forall y \in \left[m,M\right]$
\par \quad \ \romannumeral1. 若$y = m$,则$\exists x_{1} \in \left[a,b\right]$,$s.t. \ y = f(x_{1}) \in f(\left[a,b\right])$
\par \quad \ \romannumeral2. 若$y = M$,则$\exists x_{2} \in \left[a,b\right]$,$s.t. \ y = f(x_{2}) \in f(\left[a,b\right])$
\par \quad \ \romannumeral3. 若$y \in \left[m,M\right]$,由介值性定理可知,$\exists x \in [x_{1},x_{2}]$,$s.t. \ y = f(x) \in f(\left[a,b\right])$
\par \quad 所以 $\left[m,M\right] \subset f(\left[a,b\right])$
\par 综上:$f(\left[a,b\right]) = \left[m,M\right]$
