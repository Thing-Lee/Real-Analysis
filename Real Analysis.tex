\documentclass[oneside]{book} %oneside消除奇偶页出现的空页问题,页眉也会随之改变
\usepackage[zihao=-4]{ctex} %引入中文宏包
\usepackage{ulem}%提供\uline命令引入高低一致的下划线
\usepackage{amsmath}%数学公式排版
\usepackage{amssymb}%用命令 \mathbb
\usepackage{geometry}%编辑页面边距
\usepackage{pifont}%编辑带圈数字
\usepackage{fancyhdr}%风格
\usepackage[backref]{hyperref}%超链接
\usepackage{color}%字体颜色
\usepackage{extarrows}%输入带文字的长等号
\usepackage[dvipsnames,svgnames]{xcolor}
\usepackage[strict]{changepage} % 提供一个 adjustwidth 环境
\usepackage{framed} % 实现方框效果
\usepackage{amsthm}

\hypersetup{colorlinks=true,linkcolor=cyan,anchorcolor=megenta,citecolor=yellow}
\geometry{right=2.5cm,left=2.5cm,top=2.5cm,bottom=2.5cm}
\renewcommand{\headrulewidth}{0pt}%去掉页眉分划
\pagestyle{fancy}
\setcounter{chapter}{-1}%将章节编号变为从0开始


% ------------------******-------------------
\newtheorem{lemma}{推论}[section]  %推论
% environment derived from framed.sty: see leftbar environment definition
\definecolor{lashade}{rgb}{0.97,0.96,0.99} % 方框底部颜色
% ------------------******-------------------
% 注意行末需要把空格注释掉,不然画出来的方框会有空白竖线
\newenvironment{la}{%
\def\FrameCommand{%
\hspace{1pt}%
{\color{RoyalPurple}\vrule width 2pt}%竖线颜色
{\color{lashade}\vrule width 4pt}%
\colorbox{lashade}%
}%
\MakeFramed{\advance\hsize-\width\FrameRestore}%
\noindent\hspace{-4.55pt}% disable indenting first paragraph
\begin{adjustwidth}{}{7pt}%
\vspace{2pt}\vspace{2pt}%
\normalfont %环境字体设置
}
{%
\vspace{2pt}\end{adjustwidth}\endMakeFramed%
}

\newtheorem*{example}{例}
\newtheorem{examp}{例}
\definecolor{egshade}{rgb}{0.96,0.96,0.96}% 
\newenvironment{eg}{%
\def\FrameCommand{%
\hspace{1pt}%
{\color{Gray}\vrule width 2pt}%竖线颜色
{\color{egshade}\vrule width 4pt}%
\colorbox{egshade}%
}%
\MakeFramed{\advance\hsize-\width\FrameRestore}%
\noindent\hspace{-4.55pt}% 
\begin{adjustwidth}{}{7pt}%
\vspace{2pt}\vspace{2pt}%
\normalfont %环境字体设置
}
{%
\vspace{2pt}\end{adjustwidth}\endMakeFramed%
}

\newtheorem{definition}{定义}[section]  
\newtheorem{defini}{}
\newtheorem{theorem}{定理}[section]  
\newtheorem{theo}{}  
\definecolor{tdshade}{rgb}{0.94,0.97,0.93} % 
\newenvironment{td}{%
\def\FrameCommand{%
\hspace{1pt}%
{\color{Green}\vrule width 2pt}%竖线颜色
{\color{tdshade}\vrule width 4pt}%
\colorbox{tdshade}%
}%
\MakeFramed{\advance\hsize-\width\FrameRestore}%
\noindent\hspace{-4.55pt}% 
\begin{adjustwidth}{}{7pt}%
\vspace{2pt}\vspace{2pt}%
\normalfont %
}
{%
\vspace{2pt}\end{adjustwidth}\endMakeFramed%
}


\newtheorem{warning}{注}[section]  
\definecolor{washade}{rgb}{0.99,0.95,0.94} % 方框底部颜色
\newenvironment{wa}{%
\def\FrameCommand{%
\hspace{1pt}%
{\color{LightCoral}\vrule width 2pt}%竖线颜色
{\color{washade}\vrule width 4pt}%
\colorbox{washade}%
}%
\MakeFramed{\advance\hsize-\width\FrameRestore}%
\noindent\hspace{-4.55pt}% disable indenting first paragraph
\begin{adjustwidth}{}{7pt}%
\vspace{2pt}\vspace{2pt}%
\normalfont %
}
{%
\vspace{2pt}\end{adjustwidth}\endMakeFramed%
}





% ------------------******-------------------

\begin{document} 

\tableofcontents %生成目录
\chapter{导言}
\begin{eg}
\noindent \pmb{例:$Dirichlet$函数在$Riemann$意义下不可积.} 
\[ D(x) = 
\begin{cases} 
    1 & ,\text{if } x \in  [0,1] \cap \mathbb{Q} \\ 
    0 & ,\text{if } x \in [0,1]\setminus \mathbb{Q} 
\end{cases} \]
\end{eg}
\noindent 解:
\par
上积分:
\[\overline{\int_{a}^{b}} f(x) \,dx = \lim_{\left\lVert T\right\rVert  \to 0} \sum_{i = 1}^{n} M_{i}\Delta x_{i} = 1\]
\par
下积分:
\[\underline{\int_{a}^{b}} f(x) \,dx = \lim_{\left\lVert T\right\rVert  \to 0} \sum_{i = 1}^{n} m_{i}\Delta x_{i} = 0\]
\par
对 $\forall\left\lVert T\right\rVert$, 有
\[\sum_{i = 1}^{n} w_{i}\Delta x_{i} = 1\]
\par
第一充要条件和第二充要条件都不满足
\\ \hspace*{\fill}\\
\pmb{积分与极限交换次序(一般要求一致收敛)}
\begin{eg}
\pmb{例:设$\{x_{n}\}$为$[0,1]$中全体有理数,作$[0,1]$上的函数列}
\[ f_{n}(x) = 
\begin{cases} 
    1 & ,\text{if } x \in \{r_{1},r_{2},r_{3},\dots,r_{n}\}\\ 
    0 & ,\text{if } x \in [0,1]\setminus \{r_{1},r_{2},r_{3},\dots,r_{n}\}
\end{cases} \]
\end{eg}
\noindent 解:
\par
对$\forall x_{0} \in [0,1]$
\par
\ding{172} \ 若$x_{0} = r_{i} \ (i=1,\dots,n)$
\par
\quad \ 令$\delta = min\{\left\lvert x_{0}-r_{j}\right\rvert :j = 1,\dots,n,\land j\neq i\}$
\par
\quad \ 则$\forall x \in \cup _{\circ}(x_{0};\delta ) , f_{n}(x) = 0 ,\therefore \lim \limits_{x \to x_{0}}f_{n}(x)= 0$
\par
\quad \ $x_{0}$是$f_{n}(x)$的可去间断点
\par
\ding{173} \ 若$x_{0} \neq r_{i} \ (i=1,\dots,n)$
\par
\quad \ 令$\delta = min\{\left\lvert x_{0}-r_{i}\right\rvert :i = 1,\dots,n\} > 0$
\par
\quad \ 则$\forall x \in \cup(x_{0};\delta ) , f_{n}(x) = 0 ,\therefore \lim \limits_{x \to x_{0}}f_{n}(x)= 0 = f_{n}(x_{0})$
\par
\quad \ 故$f_{n}(x)$在$x_{0}$点处连续
\par
综上,$f_{n}(x)$在$[0,1]$有界且有且只有有限个间断点
\par
故$f_{n}(x)$在$[0,1]$上$Riemann$可积


\begin{eg}
\pmb{不$Riemann$可积:}
\[ \lim_{n \to \infty} f_{n}(x) = D(x) = 
\begin{cases} 
    1 & ,\text{if } x \in  [0,1] \cap \mathbb{Q} \\ 
    0 & ,\text{if } x \in [0,1]\setminus \mathbb{Q} 
\end{cases} \]
\end{eg}
\noindent 解:
\par
对$\forall x_{0} \in [0,1]$
\par
\ding{172} \ 若$x\notin \mathbb{Q} $
\par
\quad \ 则对$\forall n , f_{n}(x) = 0 , \ \therefore \lim \limits_{n \to \infty}f_{n}(x) = 0 = D(x)$
\par
\ding{173} \ 若$x\in \mathbb{Q} $
\par
\quad \ 则$\exists N \in \mathbb{N} ^{+} , s.t. \ x = r_{N}$
\par
\quad \ 对$\forall n > N , x \in \{r_{1},\dots,r_{n}\} , \ \therefore f_{n}(x) = 1$
\par
\quad \ 故$\lim \limits_{n \to \infty}f_{n}(x) = 1 = D(x)$
\\ \hspace*{\fill}\\
故对一般收敛函数列,在$Riemann$积分意义下极限运算与积分运算不一定可交换顺序
\\
即:
\[\lim_{n \to \infty} \int_{a}^{b} f_{n}(x) \,dx \stackrel{?}{=} \int_{a}^{b} \lim_{n \to \infty}  f_{n}(x) \,dx   \]

\newpage
\pmb{$Lebesgue$积分思想简介}
\begin{eg}
\pmb{问题:求$y = f(x)$,$x \in [a,b]$与$x$轴,$x = a$,$x = b$所围的曲边梯形面积}
\end{eg}
\noindent \pmb{法1:$Riemann$积分}
\par
作定义域$[a,b]$的分割$T$,$a = x_{0} < x_{1} < \dots <x_{n} = b$
\[M_{i} = \sup \limits_{[x_{i-1},x_{i}]}f(x)\]
\[m_{i} = \inf \limits_{[x_{i-1},x_{i}]}f(x)\]
\par
上和:\[\mathcal{S}(T) = \sum \limits_{i = 1}^{n}M_{i}\Delta x_{i}\]
\par
下和:\[S(T) = \sum \limits_{i = 1}^{n}M_{i}\Delta x_{i}\]
\par
若\[\lim \limits_{\left\lVert T\right\rVert  \to 0} \mathcal{S}(T) =  \lim \limits_{\left\lVert T\right\rVert  \to 0} S(T)\]
\par
则$f$在$[a,b]$上$Riemann$可积
\par
曲边梯形面积:\[S = (R) \int_{a}^{b} f(x) \,dx \]
\\
\pmb{法2:$Lebesgue$积分} 
\par
作值域$[c,d]$的分割$T$,$c = y_{0} < y_{1} < \dots <y_{n} = d$
\par
记
\[E_{i} = \{x \in [a,b] : y_{i-1} \leqslant f(x) \leqslant y_{i}\}\]
\par
积分和$\sum \limits_{i = 1}^{n} y_{i} \times E_{i} $的长度
\par
当$\left\lVert T \right\rVert \to 0$时,$S=\lim \limits_{\left\lVert T\right\rVert  \to 0} \sum \limits_{i = 1}^{n} y_{i} \times E_{i}$
\begin{eg}
\pmb{例:$Dirichlet$函数在$Lebesgue$意义下可积}
\[ D(x) = 
\begin{cases} 
    1 & ,\text{if } x \in  [0,1] \cap \mathbb{Q} \\ 
    0 & ,\text{if } x \in [0,1]\setminus \mathbb{Q} 
\end{cases} \]
\end{eg}
\noindent 解:
\par
对$\forall$值域分割$T$:有
\[0 = y_{0} < y_{1} < \dots <y_{n} = 1\]
\par
那么
\[E_{i} = \{x \in [a,b] : y_{i-1} \leqslant f(x) \leqslant y_{i}\}\]
\par
所以
\begin{center}
$\lim \limits_{\left\lVert T\right\rVert  \to 0} \sum \limits_{i = 1}^{n} y_{i} \times E_{i}$的长度$=$ $[0,1] \cap \mathbb{Q} $的长度
\end{center}

\chapter{第一章 \ 集合}
\paragraph{
本章的新内容是无限多个集合的并与交,掌握它是学习实变函数的一项基本功
\\
$Cantor$在19实际创立“集合论”,对无限集合也以大小、多少来分
\\
实变函数论建立在实数理论和集合论基础之上,本章含有大量证明去理解集合语言}
\newpage
\section{集合的表示} 
\pmb{本节重点:集合语言表示数学问题}
\begin{eg}
\pmb{例9:}若$f(x)$在$\mathbb{R}$上定义,且在$\mathbb{R}$上有上界$M$,则\[\mathbb{R} = \underbrace{\{x : f(x) \leqslant M\}}_\text{A} = \underbrace{\{x : f(x) \leqslant M + 1\}}_\text{B}\]
\end{eg}
\noindent 证明:
\par
显然:
\[A\subset B\subset \mathbb{R}\]
\par
下证:
\[\mathbb{R}\subset A\]
\par
因为$f(x)$在$\mathbb{R}$上有上界$M$
\par
故对$\forall x \in \mathbb{R}$,有$f(x) \leqslant M$
\par
故$\mathbb{R} \subset A$
\par
故$A = B = \mathbb{R}$
\begin{eg} 
\pmb{例10:}若$f(x)$在$\left[a,b\right]$上连续,则$f(\left[a,b\right]) = \left[m,M\right]$,其中$m = \min \{f(x) : x \in \left[a,b\right]\}$,$M = \max \{f(x) : x \in \left[a,b\right]\}$
\end{eg}
\noindent 证明:
\par \ding{172} 对$\forall x \in \left[a,b\right]$,有$m \leqslant f(x) \leqslant M$
\par \quad 所以$f(\left[a,b\right] \subset \left[m,M\right])$
\par \ding{173} 对$\forall y \in \left[m,M\right]$
\par \quad \ \romannumeral1. 若$y = m$,则$\exists x_{1} \in \left[a,b\right]$,$s.t. \ y = f(x_{1}) \in f(\left[a,b\right])$
\par \quad \ \romannumeral2. 若$y = M$,则$\exists x_{2} \in \left[a,b\right]$,$s.t. \ y = f(x_{2}) \in f(\left[a,b\right])$
\par \quad \ \romannumeral3. 若$y \in \left[m,M\right]$,由介值性定理可知,$\exists x \in [x_{1},x_{2}]$,$s.t. \ y = f(x) \in f(\left[a,b\right])$
\par \quad 所以 $\left[m,M\right] \subset f(\left[a,b\right])$
\par 综上:$f(\left[a,b\right]) = \left[m,M\right]$

\newpage
\section{集合的运算}
\pmb{本节重点:集合语言进行数学运算和证明}

\begin{eg} 
\pmb{例2:}$\left(a,b\right) = \bigcup \limits_{n=1}^{\infty}\left[ a + \dfrac{1}{n},b - \dfrac{1}{n}\right]$
\par $\Rightarrow$\textcolor{red}{闭区间的无穷并不一定是闭的}
\end{eg}
\noindent 证明:
\par \ding{172} $\left(a,b\right) \supset \bigcup \limits_{n=1}^{\infty}\left[ a + \dfrac{1}{n},b - \dfrac{1}{n}\right]$显然
\par \ding{173} 对$\forall x \in \left(a,b\right) $
\par \quad 有$ x > a = \lim \limits_{n \to \infty} \left( a + \dfrac{1}{n}\right)$
\par \quad 故$ \exists N_{1} \in \mathbb{N}^{+}$,$s.t.$对$\forall n > N_{1}$,有$x > a + \dfrac{1}{n}$
\par \quad 有$ x < b = \lim \limits_{n \to \infty} \left( b - \dfrac{1}{n}\right)$
\par \quad 故$ \exists N_{2} \in \mathbb{N}^{+}$,$s.t.$对$\forall n < N_{2}$,有$x < b - \dfrac{1}{n}$
\par \quad 令$N = \max \left\{N_{1},N_{2}\right\} $
\par \quad 则对$\forall n > N$,有$a + \dfrac{1}{n} < x < b - \dfrac{1}{n}$
\par \quad 即$x \in \left[ a + \dfrac{1}{n},b - \dfrac{1}{n}\right] \subset \bigcup \limits_{n=1}^{\infty}\left[ a + \dfrac{1}{n},b - \dfrac{1}{n}\right]$
\par \quad 故$ \left(a,b\right) \subset \bigcup \limits_{n=1}^{\infty}\left[ a + \dfrac{1}{n},b - \dfrac{1}{n}\right]$
\par 综上:$\left(a,b\right) = \bigcup \limits_{n=1}^{\infty}\left[ a + \dfrac{1}{n},b - \dfrac{1}{n}\right]$

\begin{eg}
    \pmb{例3:}若$Q_{n} = \left\{\dfrac{m}{n} : m \in \mathbb{Z}\right\} $,$n = 1,2,\dots$,则$\mathbb{Q} = \bigcup \limits_{n=1}^{\infty} Q_{n}$
\end{eg}
\noindent 证明:
\par \ding{172} $\mathbb{Q} \supset \bigcup \limits_{n=1}^{\infty} Q_{n}$显然
\par \ding{172} 对$\forall x \in \mathbb{Q}$
\par \quad 有$x = \dfrac{p}{q} \in \mathbb{Q}_{q} \subset \bigcup \limits_{n=1}^{\infty} Q_{n}$
\par \quad 故$ \mathbb{Q} \subset \bigcup \limits_{n=1}^{\infty} Q_{n}$
\par 综上:$\mathbb{Q} = \bigcup \limits_{n=1}^{\infty} Q_{n}$

\begin{eg}
    \pmb{例5:}若$f(x)$是定义在$E$上的函数,则$\left\{x : f(x) > 0\right\} = \bigcup \limits_{n=1}^{\infty}\left\{x : f(x) > \dfrac{1}{n}\right\} $
\end{eg}
\noindent 证明1:
\par 设$x_{0} \in \left\{x : f(x) > 0\right\}$
\par $\Leftrightarrow$则$f(x_{0}) > 0$,$\dfrac{1}{f(x_{0})} > 0$
\par $\Leftrightarrow$ \ $\exists n > \dfrac{1}{f(x_{0})}$,$f(x_{0}) > \dfrac{1}{n}$
\par $\Leftrightarrow$ \ $x_{0} \in \bigcup \limits_{n=1}^{\infty}\left\{x : f(x) > \dfrac{1}{n}\right\} $
\\
证明2:
\[\underbrace{\left\{x : f(x) > 0\right\} }_\text{$A$}= \bigcup \limits_{n=1}^{\infty}\underbrace{\left\{x : f(x) > \dfrac{1}{n}\right\}}_\text{$A_{n}$} = \bigcup \limits_{n=1}^{\infty}\underbrace{\left\{x : f(x) \geqslant \dfrac{1}{n}\right\}}_\text{$B_{n}$} \]
\par \ding{172} $\bigcup \limits_{n=1}^{\infty} A_{n} \subset \bigcup \limits_{n=1}^{\infty} B_{n} \subset A$显然
\par \ding{173} 下证:$A \subset \bigcup \limits_{n=1}^{\infty} A_{n}$
\par \quad \ 对$\forall x \in A$
\par \quad \ 有$ f(x) > 0 = \lim \limits_{n \to \infty} \dfrac{1}{n} $
\par \quad \ 故$ \exists N > 0$,对$\forall n > N$,有$f(x) > \dfrac{1}{n}$
\par \quad \ 故对$\forall n > N$,有$x \in A_{n}$
\par \quad \ 故$ x \in \bigcup \limits_{n=1}^{\infty} A_{n}$
\par \quad \ 故$ A \subset \bigcup \limits_{n=1}^{\infty} A_{n}$
\par 综上:$A = \bigcup \limits_{n=1}^{\infty} A_{n} = \bigcup \limits_{n=1}^{\infty} B_{n} $

\begin{eg}
    \pmb{例8:}若$\left\{f_{n}(x)\right\} $是定义在$E$上的一列函数,则对$\forall c \in \mathbb{R}$
\par (1)$\left\{x : \sup f_{n}(x) \leqslant c\right\} = \bigcap \limits_{n=1}^{\infty} \left\{x : f_{n}(x) \leqslant c\right\} $ \quad $\xrightarrow{regard \ as} \ A = \bigcap \limits_{n=1}^{\infty}A_{n}$
\par (2)$\left\{x : \sup f_{n}(x) > c\right\} = \bigcup \limits_{n=1}^{\infty} \left\{x : f_{n}(x) > c\right\} $ \quad $\xrightarrow{regard \ as} \ B = \bigcup \limits_{n=1}^{\infty}B_{n}$
\end{eg}
\noindent 证明:
\par (1) 
\par \ding{172} \ 对$\forall x \in A$
\par \quad \ 有$\sup f_{n}(x) \leqslant c$
\par \quad \ 故对$\forall n \in \mathbb{R}$,有$f_{n}(x) \leqslant \sup f_{n}(x) \leqslant c$
\par \quad \ 故对$\forall n \in \mathbb{R}$,有$x \in A_{n}$
\par \quad \ 故$ x \in \bigcap \limits_{n=1}^{\infty}A_{n}$
\par \quad \ 故$ A \subset \bigcap \limits_{n=1}^{\infty}A_{n}$
\par \ding{173} \ 对$\forall x \in \bigcap \limits_{n=1}^{\infty}A_{n}$
\par \quad \ 故对$\forall n \in \mathbb{R}$,有$x \in A_{n}$
\par \quad \ 故对$\forall n \in \mathbb{R}$,有$f_{n}(x) \leqslant c$
\par \quad \ 故$ \sup f_{n}(x) \leqslant c$
\par \quad \ 故$ x \in A$
\par \quad \ 故$ \bigcap \limits_{n=1}^{\infty}A_{n} \subset A$
\par 综上:$A = \bigcap \limits_{n=1}^{\infty}A_{n}$
\par (2) 
\par \ding{172} \ 对$\forall x \in B$
\par \quad \ 有$\sup f_{n}(x) > c$
\par \quad \ 故$\exists n \in \mathbb{N}^{+}$,$s.t. \ f_{n}(x) > c$
\par \quad \ 若不然
\par \quad \ 则$\exists n \in \mathbb{N}^{+}$,$s.t. \ \sup f_{n}(x) \leqslant c$,$s.t. \ f_{n}(x) \leqslant c$ 
\par \quad \ 与题设矛盾,所以必然存在$n$
\par \quad \ 故$\exists n \in \mathbb{N}^{+}$,$s.t. \ x \in B_{n}$
\par \quad \ 故$x \in \bigcup \limits_{n=1}^{\infty}B_{n}$
\par \quad \ 故$B \subset \bigcup \limits_{n=1}^{\infty}B_{n}$
\par \ding{173} \ 对$\forall x \in \bigcup  \limits_{n=1}^{\infty}B_{n}$
\par \quad \ 故$\exists n \in \mathbb{N}^{+}$,$s.t. \ x \in B_{n}$
\par \quad \ 故$\exists n \in \mathbb{N}^{+}$,$s.t. \ f_{n}(x) > c$
\par \quad \ 故$sup f_{n}(x) \geqslant  f_{n}(x) > c$
\par \quad \ 故$x \in B$
\par \quad \ 故$\bigcup \limits_{n=1}^{\infty}B_{n} \subset B$
\par 综上:$B = \bigcup \limits_{n=1}^{\infty}B_{n}$

\newpage
\begin{td}
\pmb{$De \ Morgan$公式}
\par (1) $\left( \bigcup \limits_{\alpha \in \varLambda}A_{\alpha}\right)^{c} = \bigcap \limits_{\alpha \in \varLambda}A_{\alpha}^{c}$
\par (2) $\left( \bigcap \limits_{\alpha \in \varLambda}A_{\alpha}\right)^{c} = \bigcup \limits_{\alpha \in \varLambda}A_{\alpha}^{c}$
\end{td}
\noindent 证明(2):
\par \ding{172} 对$\forall x \in \left( \bigcap \limits_{\alpha \in \varLambda}A_{\alpha}\right)^{c} $
\par \quad 则$x \notin \bigcap \limits_{\alpha \in \varLambda}A_{\alpha}$
\par \quad 故$\exists \alpha \in \varLambda $,$s.t. \ x \notin A_{\alpha}$
\par \quad 故$\exists \alpha \in \varLambda $,$s.t. \ x \in A_{\alpha}^{c}$
\par \quad 故$x \in \bigcup \limits_{\alpha \in \varLambda}A_{\alpha}^{c}$
\par \quad 故$\left( \bigcap \limits_{\alpha \in \varLambda}A_{\alpha}\right)^{c} \subset  \bigcup \limits_{\alpha \in \varLambda}A_{\alpha}^{c}$
\par \ding{173} 对$\forall x \in \bigcup \limits_{\alpha \in \varLambda}A_{\alpha}^{c}$
\par \quad 则$\exists \alpha \in \varLambda $,$s.t. \ x \in A_{\alpha}^{c}$
\par \quad 故$\exists \alpha \in \varLambda $,$s.t. \ x \notin A_{\alpha}$
\par \quad 故$x \notin \bigcap \limits_{\alpha \in \varLambda}A_{\alpha}$
\par \quad 故$x \in \left( \bigcap \limits_{\alpha \in \varLambda}A_{\alpha}\right)^{c}$
\par \quad 故$\bigcup \limits_{\alpha \in \varLambda}A_{\alpha}^{c} \subset \left( \bigcap \limits_{\alpha \in \varLambda}A_{\alpha}\right)^{c} $
\par 综上:$\left( \bigcap \limits_{\alpha \in \varLambda}A_{\alpha}\right)^{c} = \bigcup \limits_{\alpha \in \varLambda}A_{\alpha}^{c}$
\\ \hspace*{\fill}\\
\pmb{$De \ Morgan$公式的应用}

\begin{eg}
   \pmb{例11:}若$\left\{f_{n}(x)\right\}$是定义在$E$上的函数列,则:
\par (1)$\left\{ x : \left\{f_{n}(x)\right\} is \ bounded \right\} = \bigcup \limits_{M \in \mathbb{R}}\bigcap \limits_{n=1}^{\infty}\left\{x : \left\lvert f_{n}(x)\right\rvert \leqslant M\right\} $
\par (2) $\left\{ x : \left\{f_{n}(x)\right\} isn't \ bounded \right\} = \bigcap \limits_{M \in \mathbb{R}}\bigcup \limits_{n=1}^{\infty}\left\{x : \left\lvert f_{n}(x)\right\rvert > M\right\} $  
\end{eg}
\noindent 证明:
\par (1) $\xrightarrow{regard \ as} A = \bigcup \limits_{M \in \mathbb{R}}\bigcap \limits_{n=1}^{\infty}A_{n,M}$
\par \quad \ $x \in A$
\par \quad \ $\Leftrightarrow \left\{f_{n}(x)\right\}$有界
\par \quad \ $\Leftrightarrow \exists M > 0$,$s.t.$对$\forall n \in \mathbb{N}^{+}$,有$\left\lvert f_{n}(x)\right\rvert \leqslant M$
\par \quad \ $\Leftrightarrow \exists M > 0$,$s.t.$对$\forall n \in \mathbb{N}^{+}$,有$x \in A_{n,M}$
\par \quad \ $\Leftrightarrow \exists M > 0$,$s.t. \ x \in \bigcap \limits_{n=1}^{\infty}A_{n,M}$
\par \quad \ $\Leftrightarrow x \in \bigcup \limits_{M \in \mathbb{R}} \bigcap \limits_{n=1}^{\infty}A_{n,M}$
\par \quad 故 $A = \bigcup \limits_{M \in \mathbb{R}}\bigcap \limits_{n=1}^{\infty}A_{n,M}$
\par (2) $\xrightarrow{regard \ as} A = \bigcap \limits_{M \in \mathbb{R}}\bigcup \limits_{n=1}^{\infty}A_{n,M}$
\par \quad \ $De \ Morgan$公式:
\par \quad \ $\left\{ x : \left\{f_{n}(x)\right\} isn't \ bounded \right\} = \left\{ x : \left\{f_{n}(x)\right\} is \ bounded \right\}^{c}$
\par \quad \ $= \left(\bigcup \limits_{M \in \mathbb{R}}\bigcap \limits_{n=1}^{\infty}A_{n,M}\right)^{c} = \bigcap \limits_{M \in \mathbb{R}}\left(\bigcap \limits_{n=1}^{\infty}A_{n,M}\right)^{c} $
\par \quad \ $=\bigcap \limits_{M \in \mathbb{R}}\bigcup \limits_{n=1}^{\infty}A_{n,M}^{c} = \bigcap \limits_{M \in \mathbb{R}}\bigcup \limits_{n=1}^{\infty}\left\{x : \left\lvert f_{n}(x)\right\rvert > M\right\}$

\begin{eg}
\pmb{例12:}若$\left\{f_{n}(x)\right\}$是定义在$E$上的函数列,有:
\par (1) $\left\{x : \lim \limits_{n \to \infty} f_{n}(x) = 0 \right\} = \bigcap \limits_{\varepsilon \in \mathbb{R}^{+}}\bigcup \limits_{N = 1}^{\infty}\bigcap \limits_{n=N}^{\infty}\left\{x : \left\lvert f_{n}(x)\right\rvert < \varepsilon \right\} $
\par (2) $\left\{x : \lim \limits_{n \to \infty} f_{n}(x) \neq  0 \ or \ \nexists \right\} = \bigcup \limits_{\varepsilon \in \mathbb{R}^{+}}\bigcap \limits_{N = 1}^{\infty}\bigcup \limits_{n=N}^{\infty}\left\{x : \left\lvert f_{n}(x)\right\rvert \geqslant \varepsilon \right\} $
\end{eg}
\noindent 证明:
\par (1)$\xrightarrow{regard \ as} \ A = \bigcap \limits_{\varepsilon \in \mathbb{R}^{+}}\bigcup \limits_{N = 1}^{\infty}\bigcap \limits_{n=N}^{\infty}A_{n,\varepsilon }$
\par \quad \ $x \in A$
\par \quad \ $\Leftrightarrow \lim \limits_{n \to \infty} f_{n}(x) = 0$
\par \quad \ $\Leftrightarrow$对$\forall \varepsilon > 0$,$\exists N \in \mathbb{N}^{+}$,对$\forall n\geqslant N$,有$\left\lvert f_{n}(x)\right\rvert < \varepsilon$
\par \quad \ $\Leftrightarrow$对$\forall \varepsilon > 0$,$x \in \bigcup \limits_{N = 1}^{\infty}\bigcap \limits_{n=N}^{\infty}A_{n,\varepsilon }$
\par \quad \ $\Leftrightarrow x \in  \bigcap \limits_{\varepsilon \in \mathbb{R}^{+}}\bigcup \limits_{N = 1}^{\infty}\bigcap \limits_{n=N}^{\infty}A_{n,\varepsilon }$
\par \quad 故$A = \bigcap \limits_{\varepsilon \in \mathbb{R}^{+}}\bigcup \limits_{N = 1}^{\infty}\bigcap \limits_{n=N}^{\infty}A_{n,\varepsilon }$
\par (2) $De \ Morgan$公式:
\par \quad \ $\left\{x : \lim \limits_{n \to \infty} f_{n}(x) \neq  0 \ or \ \nexists \right\} = A^{c}$
\par \quad \ $= \left(\bigcap \limits_{\varepsilon \in \mathbb{R}^{+}}\bigcup \limits_{N = 1}^{\infty}\bigcap \limits_{n=N}^{\infty}A_{n,\varepsilon}\right)^{c} $
\par \quad \ $= \bigcup \limits_{\varepsilon \in \mathbb{R}^{+}}\left(\bigcup \limits_{N = 1}^{\infty}\bigcap \limits_{n=N}^{\infty}A_{n,\varepsilon}\right)^{c} $
\par \quad \ $= \bigcup \limits_{\varepsilon \in \mathbb{R}^{+}}\bigcap \limits_{N = 1}^{\infty}\left(\bigcap \limits_{n=N}^{\infty}A_{n,\varepsilon}\right)^{c} $
\par \quad \ $= \bigcup \limits_{\varepsilon \in \mathbb{R}^{+}}\bigcap \limits_{N = 1}^{\infty}\bigcup \limits_{n=N}^{\infty}A_{n,\varepsilon}^{c} $
\par \quad \ $= \bigcup \limits_{\varepsilon \in \mathbb{R}^{+}}\bigcap \limits_{N = 1}^{\infty}\bigcup \limits_{n=N}^{\infty}\left\{x : \left\lvert f_{n}(x)\right\rvert \geqslant \varepsilon \right\}$

\begin{td}
\pmb{集列上、下极限}
$\overline{\lim \limits_{n \to \infty}}A_{n}$
$\lim \limits_{\overline{n \to \infty}}A_{n}$
\[\overline{\lim \limits_{n \to \infty}}A_{n} = \left\{x : \forall N > 0,\exists n > N,s.t. \ x \in A_{n}\right\} \]    
\end{td}
\noindent 证明:
$\xrightarrow{regard \ as} \ \overline{\lim \limits_{n \to \infty}}A_{n} = A$
\par \ding{172}对$\forall x \in \overline{\lim \limits_{n \to \infty}}A_{n}$,下证:$x \in A$
\par \ 反证法:假设$x \notin A$
\par \quad \quad \quad \quad \ 则$\exists N > 0$,对$\forall n > N$,有$x \notin A_{n}$
\par \quad \quad \quad \quad \ 故$x$至多属于有限个$A_{n}$
\par \quad \quad \quad \quad \ 所以$x \notin \overline{\lim \limits_{n \to \infty}}A_{n} $ \ 矛盾
\par \quad \quad \quad \quad \ 故假设不成立
\par \quad \quad \quad \quad \ 故$x \in A$,$\overline{\lim \limits_{n \to \infty}}A_{n} \subset A$
\par \ding{173}对$\forall x \in A$,下证:$x \in \overline{\lim \limits_{n \to \infty}}A_{n}$
\par \ 反证法:假设$x \notin \overline{\lim \limits_{n \to \infty}}A_{n}$
\par \quad \quad \quad \quad \ 则$x$至多属于有限个$A_{n}$,记最大下标为$N$
\par \quad \quad \quad \quad \ 则$\forall n > N$,$x \notin A_{n}$,矛盾
\par \quad \quad \quad \quad \ 故假设不成立
\par \quad \quad \quad \quad \ 故$x \in \overline{\lim \limits_{n \to \infty}}A_{n}$,$A \subset \overline{\lim \limits_{n \to \infty}}A_{n}$
\par 综上:$\overline{\lim \limits_{n \to \infty}}A_{n} = A$
\\ \hspace*{\fill}\\
\pmb{求集列上、下极限}

\begin{eg}
\pmb{例13:}设$A_{n}$是如下一列点集:
\[ \quad \quad \quad \ A_{2m + 1} = \left[0,2 -\dfrac{1}{2m+1}\right],m=0,1,2,\dots\]
\[A_{2m} = \left[0,1 + \dfrac{1}{2m}\right],m=1,2,\dots\]
求$\lim \limits_{\overline{n \to \infty}}A_{n}$和$\overline{\lim \limits_{n \to \infty}}A_{n}$    
\end{eg}
\noindent 解:
\par 因为对$\forall n$,有$\left[0,1\right] \subset A_{n} \subset \left[0,2\right) $
\par 故有$\left[0,1\right] \subset \lim \limits_{\overline{n \to \infty}}A_{n} \subset \overline{\lim \limits_{n \to \infty}}A_{n}\subset \left[0,2\right) $
\par \ding{172} 下证:$\lim \limits_{\overline{n \to \infty}}A_{n} \subset \left[0,1\right]$
\par \quad \ 对$\forall x \in \lim \limits_{\overline{n \to \infty}}A_{n}$,$\exists N > 0$,对$\forall n > N$,$s.t. \ x \in A_{n}$
\par \quad \ 故对$\forall n > N$,$s.t. \ x \in A_{2n}$
\par \quad \ 故对$\forall n > N$,$s.t. \ 0 \leqslant  x \leqslant 1 + \dfrac{1}{2n}$
\par \quad \ 令$n \to \infty$,得$0 \leqslant x \leqslant 1$
\par \quad \ 故$x \in \left[0,1\right] $
\par \quad \ 故$\lim \limits_{\overline{n \to \infty}}A_{n} \subset \left[0,1\right]$
\par \ding{173} 下证:$\left[0,2\right) \subset \overline{\lim \limits_{n \to \infty}}A_{n}$
\par \quad \ 对$\forall x \in \left[0,2\right) $
\par \quad \ 有$x < 2 = \lim \limits_{n \to \infty} \left( 2 -\dfrac{1}{2n+1} \right)$
\par \quad \ 故$\exists N >0$,对$\forall n > N$,有$0 \leqslant  x < 2 -\dfrac{1}{2n+1} $
\par \quad \ 对$\forall n > N$,有$x \in A_{2n + 1}$
\par \quad \ 故$x \in \overline{\lim \limits_{n \to \infty}}A_{n}$
\par \quad \ 故$\left[0,2\right) \subset \overline{\lim \limits_{n \to \infty}}A_{n}$
\par 综上:$\lim \limits_{\overline{n \to \infty}}A_{n} = \left[0,1\right]$ \quad $\overline{\lim \limits_{n \to \infty}}A_{n} = \left[0,2\right)$

\begin{eg}
\pmb{例14:}设$\lim \limits_{n \to \infty} a_{n} = a$,求证:
\[\left\{a\right\} = \bigcap \limits_{\varepsilon \in \mathbb{R}^{+}} \lim \limits_{\overline{n \to \infty}}\left\{x : \left\lvert x - a_{n} \right\rvert < \varepsilon \right\} \]
\end{eg}
\noindent 证明:
$\xrightarrow{regard \ as} \ \left\{a\right\} = \bigcap \limits_{\varepsilon \in \mathbb{R}^{+}} \lim \limits_{\overline{n \to \infty}}A_{n,\varepsilon } $
\par \ $\lim \limits_{n \to \infty} a_{n} = a$
\par \ $\Leftrightarrow$对$\forall \varepsilon > 0$,$\exists N > 0$,对$\forall n > N$,有$\left\lvert x - a_{n} \right\rvert  < \varepsilon$
\par \ $\Leftrightarrow$对$\forall \varepsilon > 0$,$\exists N > 0$,对$\forall n > N$,有$a \in A_{n,\varepsilon }$
\par \ $\Leftrightarrow$对$\forall \varepsilon > 0$,有$a \in \lim \limits_{\overline{n \to \infty}}A_{n,\varepsilon }$
\par \ $\Leftrightarrow$$a \in \bigcap \limits_{\varepsilon \in \mathbb{R}^{+}} \lim \limits_{\overline{n \to \infty}}A_{n,\varepsilon }$
\par 由极限的唯一性得:$\left\{a\right\} = \bigcap \limits_{\varepsilon \in \mathbb{R}^{+}} \lim \limits_{\overline{n \to \infty}}\left\{x : \left\lvert x - a_{n} \right\rvert < \varepsilon \right\} $
\begin{td}
\pmb{Theorem \ 3 :}
\[\overline{\lim \limits_{n \to \infty}}A_{n} = \bigcap \limits_{n=1}^{\infty}\bigcup \limits_{m=n}^{\infty}A_{m}\]
\[\lim \limits_{\overline{n \to \infty}}A_{n} = \bigcup \limits_{n=1}^{\infty}\bigcap \limits_{m=n}^{\infty}A_{m}\]
\end{td}
\noindent (1)
\par \ding{172} 先证:$\overline{\lim \limits_{n \to \infty}}A_{n} \subset  \bigcap \limits_{n=1}^{\infty}\bigcup \limits_{m=n}^{\infty}A_{m}$
\par \quad 对$\forall x \in \overline{\lim \limits_{n \to \infty}}A_{n} = \left\{x : \forall N > 0,\exists m > N,s.t. \ x \in A_{m}\right\} \dots \dots \dots \dots(\romannumeral1)$
\par \quad 对$\forall n \in \mathbb{N}^{+}$,由$(\romannumeral1)$式知,$\exists m > n$,$s.t. \ x \in A_{m}$
\par \quad 对$\forall n \in \mathbb{N}^{+}$,有$x \in \bigcup \limits_{m=n}^{\infty}A_{m}$
\par \quad 故$x \in \bigcap \limits_{n=1}^{\infty}\bigcup \limits_{m=n}^{\infty}A_{m}$
\par \quad 故$\overline{\lim \limits_{n \to \infty}}A_{n} \subset  \bigcap \limits_{n=1}^{\infty}\bigcup \limits_{m=n}^{\infty}A_{m}$
\par \ding{173} 下证:$ \bigcap \limits_{n=1}^{\infty}\bigcup \limits_{m=n}^{\infty}A_{m} \subset \overline{\lim \limits_{n \to \infty}}A_{n}$
\par \quad 对$\forall x \in \bigcap \limits_{n=1}^{\infty}\bigcup \limits_{m=n}^{\infty}A_{m}$
\par \quad 则对$\forall n \in \mathbb{N}^{+}$,$\exists m > n$,$s.t. \ x \in A_{m} \dots \dots \dots \dots(\romannumeral2)$
\par \quad 对$\forall N > 0$,令$n_{0} = \left[N\right] + 1 \in \mathbb{N}^{+} $
\par \quad 由$(\romannumeral2)$式知,$\exists m \geqslant n_{0} > N$,$s.t. \ x \in A_{m}$
\par \quad 故$x \in \overline{\lim \limits_{n \to \infty}}A_{n}$
\par \quad 故$\bigcap \limits_{n=1}^{\infty}\bigcup \limits_{m=n}^{\infty}A_{m} \subset \overline{\lim \limits_{n \to \infty}}A_{n}$
\par 综上:$\overline{\lim \limits_{n \to \infty}}A_{n} = \bigcap \limits_{n=1}^{\infty}\bigcup \limits_{m=n}^{\infty}A_{m}$
\\(2)
\par \ding{172} 先证:$\lim \limits_{\overline{n \to \infty}}A_{n} \subset  \bigcup \limits_{n=1}^{\infty}\bigcap \limits_{m=n}^{\infty}A_{m}$
\par \quad 对$\forall x \in \lim \limits_{\overline{n \to \infty}}A_{n} = \left\{x : \exists N > 0,\forall m > N,s.t. \ x \in A_{m}\right\} \dots \dots \dots \dots(\romannumeral3)$
\par \quad 令$n_{0} = \left[N\right] + 1 \in \mathbb{N}^{+}$
\par \quad 由$(\romannumeral3)$式知,对$\forall m \geqslant n_{0} > N$,有$x \in A_{m}$
\par \quad 则$x \in \bigcap \limits_{m=n}^{\infty}A_{m}$
\par \quad 故$x \in \bigcup \limits_{n=1}^{\infty}\bigcap \limits_{m=n}^{\infty}A_{m}$
\par \quad 故$\lim \limits_{\overline{n \to \infty}}A_{n} \subset  \bigcup \limits_{n=1}^{\infty}\bigcap \limits_{m=n}^{\infty}A_{m}$
\par \ding{173} 下证:$ \bigcup \limits_{n=1}^{\infty}\bigcap \limits_{m=n}^{\infty}A_{m} \subset \lim \limits_{\overline{n \to \infty}}A_{n}$
\par \quad 对$\forall x \in \bigcup \limits_{n=1}^{\infty}\bigcap \limits_{m=n}^{\infty}A_{m} $
\par \quad 可知$\exists n \in \mathbb{N}^{+}$,对$\forall m \geqslant n$,有$x \in A_{m} \dots \dots \dots \dots(\romannumeral4)$
\par \quad 令$N = n > 0$
\par \quad 由$(\romannumeral4)$式知,对$\forall m > N$,有$x \in A_{m}$
\par \quad 故$x \in \lim \limits_{\overline{n \to \infty}}A_{n}$
\par \quad 故$ \bigcup \limits_{n=1}^{\infty}\bigcap \limits_{m=n}^{\infty}A_{m} \subset \lim \limits_{\overline{n \to \infty}}A_{n}$
\par 综上:$\lim \limits_{\overline{n \to \infty}}A_{n} = \bigcup \limits_{n=1}^{\infty}\bigcap \limits_{m=n}^{\infty}A_{m}$

\begin{eg}
\pmb{例:}
\par (1)$\left\{x : \lim \limits_{n \to \infty}f_{n}(x) = 0\right\} = \bigcap \limits_{\varepsilon  \in \mathbb{R}^{+}} \lim \limits_{\overline{n \to \infty}}\left\{x : \left\lvert f_{n}(x)\right\rvert < \varepsilon \right\}$
\par (2)$\left\{x : \lim \limits_{n \to \infty}f_{n}(x) \neq 0 \ or \ \nexists \right\} = \bigcup  \limits_{\varepsilon  \in \mathbb{R}^{+}} \overline{\lim \limits_{n \to \infty}}\left\{x : \left\lvert f_{n}(x)\right\rvert \geqslant \varepsilon \right\}$    
\end{eg}
\noindent 证明:
\\(1)
\par $\left\{x : \lim \limits_{n \to \infty}f_{n}(x) = 0\right\} \xlongequal{eg.12} \bigcap \limits_{\varepsilon  \in \mathbb{R}^{+}}\bigcup \limits_{N=1}^{\infty}\bigcap \limits_{n=N}^{\infty}\left\{x : \left\lvert f_{n}(x)\right\rvert < \varepsilon \right\}$
\par $\xlongequal{Th.3} \bigcap \limits_{\varepsilon  \in \mathbb{R}^{+}} \lim \limits_{\overline{n \to \infty}}\left\{x : \left\lvert f_{n}(x)\right\rvert < \varepsilon \right\}$
\\(2)
\par $\left\{x : \lim \limits_{n \to \infty}f_{n}(x) \neq 0 \ or \ \nexists \right\} \xlongequal{eg.12} \bigcup \limits_{\varepsilon  \in \mathbb{R}^{+}}\bigcap \limits_{N=1}^{\infty}\bigcup \limits_{n=N}^{\infty}\left\{x : \left\lvert f_{n}(x)\right\rvert \geqslant  \varepsilon \right\}$
\par $\xlongequal{Th.3} \bigcup  \limits_{\varepsilon  \in \mathbb{R}^{+}} \overline{\lim \limits_{n \to \infty}}\left\{x : \left\lvert f_{n}(x)\right\rvert \geqslant \varepsilon \right\}$
\begin{eg}
\pmb{例:} 若$A_{n} = \left[0, 1 + \dfrac{1}{n}\right] ,n = 1,2,3,\dots$,则$\lim \limits_{n \to \infty} A_{n} = \left[0,1\right] $    
\end{eg}
\noindent 证明:
\par 对$\forall n \in \mathbb{Z}^{+}$,有$\left[0,1\right] \subset A_{n}$
\par 则有$\left[0,1\right] \subset \lim \limits_{\overline{n \to \infty}}A_{n} \subset \overline{\lim \limits_{n \to \infty}}A_{n}$
\par 下证:$\overline{\lim \limits_{n \to \infty}}A_{n} \subset \left[0,1\right]$
\par 对$\forall x \in \overline{\lim \limits_{n \to \infty}}A_{n} $,$x \geqslant 0$显然
\par 下证:$x \leqslant 1$
\par 反证法:假设$x >1 = \lim \limits_{n \to \infty} \left(1 + \dfrac{1}{n}\right)$
\par \quad \quad \quad \quad 则$\exists N > 0$,对$\forall n > N$,有$x > 1 + \dfrac{1}{n}$
\par \quad \quad \quad \quad 故对$\forall n > N$,有$x \notin A_{n}$
\par \quad \quad \quad \quad 故$x$至多属于有限个$A_{n}$
\par \quad \quad \quad \quad 则$x \notin \overline{\lim \limits_{n \to \infty}}A_{n}$,矛盾
\par \quad \quad \quad \quad 故假设不成立
\par \quad \quad \quad \quad 故$x \leqslant 1$,即$x \in \left[0,1\right] $
\par \quad \quad \quad \quad 故$\overline{\lim \limits_{n \to \infty}}A_{n} \subset \left[0,1\right]$
\par 综上:$\lim \limits_{\overline{n \to \infty}}A_{n} = \overline{\lim \limits_{n \to \infty}}A_{n} = \left[0,1\right]$
\par \quad \quad \quad 即$\left\{A_{n}\right\}$收敛且$\lim \limits_{n \to \infty} A_{n} = \left[0,1\right] $
\newpage
\begin{wa}
\pmb{注:单调集列是收敛的}
\par (1)若$\left\{A_{n}\right\}$单调递增,则$\lim \limits_{n \to \infty} A_{n} = \bigcup \limits_{n=1}^{\infty}A_{n}$
\par (1)若$\left\{A_{n}\right\}$单调递减,则$\lim \limits_{n \to \infty} A_{n} = \bigcap \limits_{n=1}^{\infty}A_{n}$
\end{wa}
\noindent 证明:
\par (1)
\par $\left\{A_{n}\right\}$递增
\par $\lim \limits_{\overline{n \to \infty}}A_{n} \xlongequal{Th.3}\bigcup \limits_{n=1}^{\infty}\bigcap \limits_{m=n}^{\infty}A_{m} = \bigcup \limits_{n=1}^{\infty}A_{n}$
\par $\overline{\lim \limits_{n \to \infty}}A_{n} \xlongequal{Th.3}\bigcap \limits_{n=1}^{\infty}\underbrace{\bigcup \limits_{m=n}^{\infty}A_{m}}_\text{$B_{n}$} \subset B_{1} = \bigcup \limits_{m=1}^{\infty}A_{m} = \bigcup \limits_{n=1}^{\infty}A_{n}$
\par 故$\bigcup \limits_{n=1}^{\infty}A_{n} = \lim \limits_{\overline{n \to \infty}}A_{n} \subset \overline{\lim \limits_{n \to \infty}}A_{n} \subset \bigcup \limits_{n=1}^{\infty}A_{n}$
\par 即$\lim \limits_{\overline{n \to \infty}}A_{n} = \overline{\lim \limits_{n \to \infty}}A_{n} = \bigcup \limits_{n=1}^{\infty}A_{n}$
\par 故$\left\{A_{n}\right\}$收敛且$\lim \limits_{n \to \infty} A_{n} = \bigcup \limits_{n=1}^{\infty}A_{n}$
\par (2)
\par $\left\{A_{n}\right\}$递减
\par $\overline{\lim \limits_{n \to \infty}}A_{n} \xlongequal{Th.3}\bigcap \limits_{n=1}^{\infty}\bigcup \limits_{m=n}^{\infty}A_{m} = \bigcap \limits_{n=1}^{\infty}A_{n}$
\par $\lim \limits_{\overline{n \to \infty}}A_{n} \xlongequal{Th.3}\bigcup \limits_{n=1}^{\infty}\underbrace{\bigcap \limits_{m=n}^{\infty}A_{m}}_\text{$B_{n}$} \supset  B_{1} = \bigcap \limits_{m=1}^{\infty}A_{m} = \bigcap \limits_{n=1}^{\infty}A_{n}$
\par 故$\bigcap \limits_{n=1}^{\infty}A_{n} \subset \lim \limits_{\overline{n \to \infty}}A_{n} \subset \overline{\lim \limits_{n \to \infty}}A_{n} = \bigcap \limits_{n=1}^{\infty}A_{n}$
\par 即$\lim \limits_{\overline{n \to \infty}}A_{n} = \overline{\lim \limits_{n \to \infty}}A_{n} = \bigcap \limits_{n=1}^{\infty}A_{n}$
\par 故$\left\{A_{n}\right\}$收敛且$\lim \limits_{n \to \infty} A_{n} = \bigcap \limits_{n=1}^{\infty}A_{n}$
\begin{eg}
\pmb{例15:}若$f_{n}(x)$是定义在$E$上的有限函数
\par (1)若$F_{n} = \left\{x : \left\lvert f(x)\right\rvert \geqslant \dfrac{1}{n}\right\} ,n=1,2,\dots$,则$\left\{F_{n}\right\}$是增加集列,且
\[\lim \limits_{n \to \infty} F_{n} = \left\{x : f(x)\neq 0\right\}\]    
\end{eg}
\noindent 证明:对$\forall n \in \mathbb{Z}^{+}$,有$F_{n}\subset F_{n+1}$
\par \quad 故$\left\{F_{n}\right\}$增加
\par \quad $\lim \limits_{n \to \infty} F_{n} = \bigcup \limits_{n=1}^{\infty}F_{n} = \bigcup \limits_{n=1}^{\infty}\left\{x : \left\lvert f(x)\right\rvert \geqslant \dfrac{1}{n}\right\}$
\par \quad $\xlongequal{eg.5}\left\{x : f(x) > 0\right\}$
\par \quad $= \left\{x : f(x)\neq 0\right\}$
\begin{eg}
\par (2)若$E_{n} = \left\{x : f(x) > n\right\} ,n=1,2,\dots$,则$\left\{E_{n}\right\}$是减少集列,且
\[\lim \limits_{n \to \infty} E_{n} = \varnothing \]    
\end{eg}
\noindent 证明:对$\forall n \in \mathbb{Z}^{+}$,有$E_{n+1}\subset E_{n}$
\par \quad 故$\left\{E_{n}\right\}$减少
\par \quad 故$\lim \limits_{n \to \infty} E_{n} = \bigcap \limits_{n=1}^{\infty}E_{n} $
\par \quad 下证:$\bigcap \limits_{n=1}^{\infty}E_{n} = \varnothing $
\par \quad 反证法:假设$\exists x \in \bigcap \limits_{n=1}^{\infty}E_{n}$
\par \quad \quad \quad \quad \quad 则对$\forall n \in \mathbb{Z}^{+}$,有$x \in E_{n}$
\par \quad \quad \quad \quad \quad 则对$\forall n \in \mathbb{Z}^{+}$,有$ f(x) > n$
\par \quad \quad \quad \quad \quad 令$n \to \infty$,得$f(x) = + \infty$,矛盾
\par \quad \quad \quad \quad \quad 故假设不成立
\par \quad \quad \quad \quad \quad 所以$\lim \limits_{n \to \infty} E_{n} = \varnothing $
\newpage
\section{对等与基数}
\pmb{本节重点:理解对等本质}
\\ \hspace*{\fill}\\
\pmb{例2:}正奇数全体与正偶数全体对等
\[\varphi :x\mapsto x+1\]
\pmb{例3:}正整数全体与正偶数全体对等
\[\varphi :x\mapsto 2x\]
\pmb{例4:}$\left(0,1\right) $与$\mathbb{R}$对等
\[\varphi :x\mapsto \tan \left(\pi x - \dfrac{\pi}{2}\right)\]
\pmb{注:}例3与例4说明无限集可以与其真子集对等

\begin{td}
\pmb{Theorem \ 1:}对任意集合$A$,$B$,$C$,均有:
\par (1)自反性:$A\thicksim A$
\par (2)对称性:$A\thicksim B \Rightarrow B\thicksim A$
\par (3)传递性:$A\thicksim B,B\thicksim C\Rightarrow A\thicksim C$    
\end{td}
\noindent 证明:
\par (1) $\varphi :x\mapsto x$
\par (2)已知$A\thicksim B$,故$\exists$双射$\varphi$,$s.t. \ A \to B$,所以有$\varphi ^{-1}$,$s.t. \ B \to A$,即$B\thicksim A$
\par (3)已知$A\thicksim B$且$B\thicksim C$
\par \quad \ 故$\exists$双射$\varphi_{1}$,$s.t. \ A \to B$且$\exists$双射$\varphi_{2}$,$s.t. \ B \to C$
\par \quad \ 令$\varphi :A \to B \to C$,则$\varphi = \varphi _{1}\circ \varphi _{2}$
\par \quad \ 易知$\varphi $是双射,故$A\thicksim C$

\begin{wa}
\pmb{\textcolor{red}{$\bigstar$注:}}
\par (1)若$A \subset B$,则$\overline{\overline{A}} \leqslant \overline{\overline{B}}$
\par (2)设$A$,$B$是两个集合,若存在$A$到$B$的单射,则$\overline{\overline{A}} \leqslant \overline{\overline{B}}$
\par (3)设$A$,$B$是两个集合,若存在$A$到$B$的满射,则$\overline{\overline{A}} \geqslant \overline{\overline{B}}$    
\end{wa}
\noindent 证明:
\par (1) (\romannumeral1)若$A\thicksim B$,则$\overline{\overline{A}} = \overline{\overline{B}}$
\par \quad \ (\romannumeral2)若$A\nsim B$,则$A \subsetneq B$,而$A \thicksim A\subsetneq B$,故$\overline{\overline{A}} < \overline{\overline{B}}$
\par (2) $\exists$单射$\varphi ,\ s.t. \ A \to B$
\par \quad \ 若$\varphi :A \to \varphi (A)$是双射
\par \quad \ 则$A \thicksim \varphi (A)$,$\overline{\overline{A}} = \overline{\overline{\varphi (A)}}$
\par \quad \ 易知$\varphi (A)\subset B$,故$\overline{\overline{\varphi (A)}} \leqslant \overline{\overline{B}}$
\par \quad \ 故$\overline{\overline{A}} \leqslant \overline{\overline{B}}$
\par (3) $\exists$满射$\varphi ,\ s.t. \ A \to B$
\par \quad \ 即对$\forall y \in B$,$\exists x \in A$,$s.t. \ y = \varphi (x)$
\par \quad \ 在$\left\{x \in A : y = \varphi (x)\right\} $中取且只取一个元素,记为$x_{y}$,令$A_{0} = \left\{x_{y} : y \in B\right\} \subset A$
\par \quad \ 易知$\varphi$是$A_{0} \to B$双射,故有$A_{0}\thicksim B$
\par \quad \ 即$\overline{\overline{B}} = \overline{\overline{A_{0}}} \leqslant \overline{\overline{A}}$

\begin{td}
\pmb{Definition \ 7:}设$A$,$B$是两个集合,如果$A$不与$B$的对等,但存在$B$的真子集$B^{*}$,有$A \thicksim B^{*}$,则称$A$比$B$有较小的基数,并记为$\overline{\overline{A}} \leqslant \overline{\overline{B}}$    
\end{td}

\begin{la}
\pmb{引理:}设$\left\{A_{\lambda } : \lambda \in \varLambda \right\}$ ,$\left\{B_{\lambda } : \lambda \in \varLambda \right\}$是两个集族,$\varLambda $是一个指标集,又$\forall \lambda  \in \varLambda $,$A_{\lambda } \thicksim B_{\lambda}$,而且$\left\{A_{\lambda } : \lambda \in \varLambda \right\}$ 中的集合两两不相交,$\left\{B_{\lambda } : \lambda \in \varLambda \right\}$中的集合两两不相交,那么
\begin{equation}
 \bigcup \limits_{\lambda \in \varLambda }A_{\lambda}\thicksim \bigcup \limits_{\lambda \in \varLambda }B_{\lambda}   \tag{le} \label{Th2}
\end{equation}
\end{la} 
\noindent 证明:
\par 因对$\forall \lambda \in \varLambda $,有$A_{\lambda } \thicksim B_{\lambda}$
\par 故对$\forall \lambda \in \varLambda $,$\exists$双射$\varphi _{\lambda } : A_{\lambda }\to B_{\lambda }$
\par 对$\forall x \in \bigcup \limits_{\lambda \in \varLambda }A_{\lambda}$,由于$\left\{A_{\lambda } \right\}$两两不相交
\par 故$\exists \lambda \in \varLambda $,$s.t. \ x \in A_{\lambda }$
\par 令$\varphi (x) = \varphi _{\lambda }(x) \in B_{\lambda }\subset \bigcup \limits_{\lambda \in \varLambda }B_{\lambda}$
\par 则有$\varphi : \bigcup \limits_{\lambda \in \varLambda }A_{\lambda}\to \bigcup \limits_{\lambda \in \varLambda }B_{\lambda}$
\par 对$\forall y \in \bigcup \limits_{\lambda \in \varLambda }B_{\lambda}$,由于$\left\{B_{\lambda }\right\} $两两不相交
\par 故$\exists \lambda \in \varLambda $,$s.t. \ y \in B_{n}$
\par 因双射$\varphi _{\lambda } : A_{\lambda }\to B_{\lambda }$
\par 故$\exists x \in A_{\lambda }\subset \bigcup \limits_{\lambda \in \varLambda }A_{\lambda}$,$s.t. \ y = \varphi _{\lambda }(x)$
\par 即$y =\varphi (x)$
\par 故$\varphi $是双射,$\bigcup \limits_{\lambda \in \varLambda }A_{\lambda}\thicksim \bigcup \limits_{\lambda \in \varLambda }B_{\lambda}$

\begin{td}
\pmb{Theorem \ 2 \ (Bernstein定理):}设$A$,$B$,是两个非空集合,若$A$对等于$B$的一个子集,$B$对等于$A$的一个子集,则$A$与$B$对等,即:
\begin{center}
    若$\overline{\overline{A}} \leqslant \overline{\overline{B}}$,$\overline{\overline{B}} \leqslant \overline{\overline{A}}$,则$\overline{\overline{A}} = \overline{\overline{B}}$ 
\end{center}    
\end{td}
\noindent 证明:
\par 已知:
$$A \thicksim B_{1}\subset  B \quad B\thicksim A_{1} \subset  A $$
\par 则:
\begin{center}
$\exists$双射$\varphi _{1} : A \to B_{1}$ ,$\exists$双射$\varphi _{2} : B \to A_{1}$ 
\end{center}
\par 令:
$$A_{2} = \varphi _{2}(B_{1})$$ 
\par 由于:
$$B_{1}\subset B$$
\par 则:
$$A_{2}\subset \varphi _{2}(B) = A_{1}$$
\par 又因:
$$\varphi _{2}: B_{1}  \stackrel{1-1}{\to} A_{2}$$
\par 故有:
$$A \stackrel{\varphi _{1}}{\thicksim }B_{1}\stackrel{\varphi _{2}}{\thicksim }A_{2}$$
\par 令:
$$B_{2} = \varphi _{1}(A_{1})$$
\par 则:
$$B_{2}\subset \varphi _{1}(A) = B_{1},\quad \varphi _{1}: A_{1}  \stackrel{1-1}{\to} B_{2}$$
\par 故有:
$$B \stackrel{\varphi _{2}}{\thicksim }A_{1}\stackrel{\varphi _{1}}{\thicksim }B_{2}$$
\par 令:
$$A_{3} = \varphi _{2}(B_{2}) \quad B_{3} = \varphi _{1}(A_{2})$$
$$A_{4} = \varphi _{2}(B_{3}) \quad B_{4} = \varphi _{1}(A_{3})$$
\par 则:
$$A\supset A_{1}\supset A_{2}\supset A_{3}\supset \dots$$
$$B\supset B_{1}\supset B_{2}\supset B_{3}\supset \dots$$
\par 且:
$$A \stackrel{\varphi _{1}}{\thicksim }B_{1}\stackrel{\varphi _{2}}{\thicksim }A_{2}\stackrel{\varphi _{1}}{\thicksim }B_{3}\stackrel{\varphi _{2}}{\thicksim }A_{4}\thicksim \dots \stackrel{\varphi _{1}}{\thicksim }B_{2n-1}\stackrel{\varphi _{2}}{\thicksim }A_{2n} \thicksim \dots$$
$$B\stackrel{\varphi _{2}}{\thicksim }A_{1}\stackrel{\varphi _{1}}{\thicksim }B_{2}\stackrel{\varphi _{2}}{\thicksim }A_{3}\stackrel{\varphi _{1}}{\thicksim }B_{4}\thicksim \dots \stackrel{\varphi _{2}}{\thicksim }A_{2n-1}\stackrel{\varphi _{1}}{\thicksim }B_{2n} \thicksim \dots $$
\par 令$\varphi =\varphi _{1}\circ \varphi _{2}$,则:
$$A \stackrel{\varphi}{\thicksim }A_{2}\stackrel{\varphi}{\thicksim }A_{4}\stackrel{\varphi}{\thicksim } \dots \stackrel{\varphi}{\thicksim}A_{2n} \thicksim \dots$$
$$A_{1} \stackrel{\varphi}{\thicksim }A_{3}\stackrel{\varphi}{\thicksim }A_{5}\stackrel{\varphi}{\thicksim } \dots \stackrel{\varphi}{\thicksim}A_{2n-1} \thicksim \dots$$
\par 故对$\forall n \in \mathbb{Z}^{+}$,$\varphi $是双射:
$$A_{2} = \varphi (A),\dots,A_{n+2} = \varphi (A_{n})$$
\par 故对$\forall n \in \mathbb{Z}^{+}$:
$$A_{2}\thicksim A,\dots,A_{n+2}\thicksim A_{n}$$
\par 故对$\forall n \in \mathbb{Z}^{+}$,有:
\begin{equation}
   A_{n}\setminus A_{n+1}\thicksim A_{n+2}\setminus A_{n+3} \tag{*} \label{6}
\end{equation}
\par 将$A$与$A_{1}$分解为互不相交的子集的并:
\begin{align*}
    A & = \left(A\setminus A_{1}\right) \cup A_{1} \\
    & = \left(A\setminus A_{1}\right) \cup \left(A_{1}\setminus A_{2}\right) \cup A_{2}\\
    & = \left(A\setminus A_{1}\right) \cup \left(A_{1}\setminus A_{2}\right) \cup \left(A_{2}\setminus A_{3}\right) \cup \dots \cup D\\
    A_{1} & = \left(A_{1}\setminus A_{2}\right) \cup A_{2} \\
    & = \left(A_{1}\setminus A_{2}\right) \cup \left(A_{2}\setminus A_{3}\right) \cup A_{3}\\
    & = \left(A_{1}\setminus A_{2}\right) \cup \left(A_{2}\setminus A_{3}\right) \cup \left(A_{3}\setminus A_{4}\right) \cup \dots \cup D_{1}
\end{align*}
\par 其中:
$$D = A\cap A_{1} \cap A_{2} \cap A_{3} \cap \dots$$
$$D_{1} = A_{1} \cap A_{2} \cap A_{3} \cap A_{4} \cap \dots$$
\par 由于:
$$A\supset A_{1}$$
\par 故有:
$$D = D_{1},\quad D\thicksim D_{1}$$
\par 由 \eqref{6}式,两式分式对等:
\begin{align*}
    A & =\left(A\setminus A_{1}\right) \cup \left(A_{1}\setminus A_{2}\right) \cup \left(A_{2}\setminus A_{3}\right) \cup \dots \cup D\\
    A_{1} & =\left(A_{1}\setminus A_{2}\right) \cup \left(A_{2}\setminus A_{3}\right) \cup \left(A_{3}\setminus A_{4}\right) \cup \dots \cup D_{1}
\end{align*}
\par 由引理知:
$$A \thicksim A_{1}$$
\par 又因$B\thicksim A_{1}$,由Th \ 1知:
$$A\thicksim B$$

\begin{la}
    \pmb{推论:}若$A\supset B\supset C$且$A \thicksim C$,则$A\thicksim B\thicksim C$   
\end{la}
\noindent 证明:
\par 已知$A \thicksim C$故:
$$\exists \varphi: A \stackrel{1-1}{\to} C$$
\par 令$C^{*} = \varphi (B) \subset \varphi (A) = C$,则:
$$\varphi: B \stackrel{1-1}{\to} C^{*}$$
\par 故:$B\thicksim C^{*}\subset C$
\par 又因:$C\thicksim C\subset B$
\par 由Th2 \tiny{\pageref{Th2} } \normalsize 知,$B \thicksim C$
\newpage
\section{可数集合}
\pmb{本节重点:掌握各个定理的区别和联系,本节需要反复学习}
\begin{td}
\pmb{定义:}与全体正整数所成的集合$\mathbb{Z}^{+}$对等的集合称为可数集合或可列集合,
\par \quad  基数记为$a$或$\aleph _{0}$    
\end{td}

\begin{wa}
\pmb{注1:}$\mathbb{Z}^{+}$可按大小排序成无穷序列:$1,2,3,\dots,n,\dots$\par
\pmb{\textcolor{red}{$\bigstar$}}$A$是可数集合$\Leftrightarrow $ \ $A$可排成无穷序列:$a_{1},a_{2},a_{3},\dots,a_{n},\dots$    
\end{wa}
\noindent 证明:
\par $''\Rightarrow''$由于$A$是可数集合,故$A\thicksim \mathbb{Z}^{+}$
\par \quad \quad \ 故$\exists \varphi : \mathbb{Z}^{+} \stackrel{1-1}{\to} A ,n \mapsto \varphi (n)$
\par \quad \quad \ 对$\forall n \in \mathbb{Z}^{+}$,记$a_{n} = \varphi (n)$
\par \quad \quad \ 则$A$可排成无穷序列$A = \varphi (\mathbb{Z}^{+}) : a_{1},a_{2},a_{3},\dots,a_{n},\dots$
\par $''\Leftarrow ''$已知$A$可排成无穷序列$A = \varphi (\mathbb{Z}^{+}) : a_{1},a_{2},a_{3},\dots,a_{n},\dots$
\par \quad \quad \ 令$\varphi : \mathbb{Z}^{+} \to A ,n \mapsto a_{n}$
\par \quad \quad \ 易知$\varphi $是双射
\par \quad \quad \ 故$\mathbb{Z}^{+}\thicksim A$,$A$是可数集合

\begin{td}
\pmb{Theorem \ 1:}任何无限集合都至少包含一个可数子集    
\end{td}
\noindent 证明:
\par 设$M$是无限集,则$M \neq \varnothing $
\par 故$\exists e_{1} \in M$
\par 易知$M \setminus \left\{e_{1}\right\} \neq \varnothing$故$\exists e_{2} \in M\setminus \left\{e_{1}\right\} $且$e_{2} \neq e_{1}$
\par 依此类推\dots \dots
\par 易知$M \setminus \left\{e_{1},e_{2},\dots,e_{n-1}\right\} \neq \varnothing$故$\exists e_{n} \in M\setminus \left\{e_{1},e_{2},\dots,e_{n-1}\right\} $且$e_{1},\dots,e_{n}$互异
\par 令$M_{0} = \left\{e_{1},e_{2},\dots,e_{n},\dots\right\} $,则$M_{0} \subset M$
\par 因$M_{0}$可排成无穷序列,故$M_{0}$是可数集,得证

\begin{wa}
\pmb{注2:}可数集在无限集中有最小的基数,即若$A$是无限集,$B$是可数集,则$\overline{\overline{A}} \geqslant \overline{\overline{B}}$    
\end{wa}
\noindent 证明:
\par 已知$A$是无限集,由Th1可知$A$存在一个可数集,记为$A^{*}$,$A^{*}\subset A$
\par 因$A^{*}$和$B$都为可数集,它们都能与$\mathbb{Z}^{+}$对等,故$A^{*}\thicksim B$
\par 故$\overline{\overline{B}} = \overline{\overline{A^{*}}} \leqslant \overline{\overline{A}}$

\begin{td}
\pmb{Theorem \ 2:}可数集合的任何\uline{无限}子集必为可数集合,从而
\par \quad \quad \quad \quad 可数集合的\uline{任意}子集或者是有限集或者是可数集    
\end{td}
\noindent 证明:
\par 设$A$是可数集合,$A^{*}$是$A$的无限子集
\par 因$A^{*}\subset A$,故$\overline{\overline{A^{*}}} \leqslant \overline{\overline{A}}$
\par 而$A^{*}$是无限集,由注2,有$\overline{\overline{A^{*}}} \geqslant \overline{\overline{A}}$
\par 由$Bernstein$定理,$\overline{\overline{A^{*}}} = \overline{\overline{A}}$
\par 故$A^{*}\thicksim A$,故$A^{*}$是可数集

\begin{td}
\pmb{Theorem \ 3:}设$A$为可数集,$B$为有限集或可数集,则$A \cup B$为可数集    
\end{td}
\noindent 证明:
\par \ding{172}当$A \cap B = \varnothing $
\par 已知$A$是可数集,记为$A = \left\{a_{1},a_{2},\dots,a_{n},\dots\right\} $
\par \romannumeral1. 若$B$是有限集,记为$B = \left\{b_{1},b_{2},\dots,b_{n}\right\} $
\par \quad 则$A \cup B = \left\{b_{1},b_{2},\dots,b_{n},a_{1},a_{2},\dots,a_{n},\dots\right\} $
\par \quad 故$A \cup B$为可数集
\par \romannumeral2. 若$B$是可数集,记为$B = \left\{b_{1},b_{2},\dots,b_{n},\dots\right\} $
\par \quad 则$A \cup B = \left\{a_{1},b_{1},a_{2},b_{2},\dots,a_{n},b_{n},\dots\right\} $
\par \quad 故$A \cup B$为可数集
\par \ding{173}当$A \cap B \neq \varnothing $
\par 令$B^{*} = B \setminus A$,则$A \cap B^{*} = \varnothing$,且$A \cup B = A \cup B^{*}$
\par 因$B^{*} \subset B$,故$B^{*}$为有限集或可数集
\par 由\ding{172}知$A \cup B^{*}$应是可数集,故$A \cup B$是可数集

\begin{la}
\pmb{推论:}设$A_{i}\left(i =1,2,\dots,n\right) $是有限集或可数集,则$\bigcup \limits_{i=1}^{n}A_{i}$也是有限集或可数集
\par \quad 若至少一个是可数集,则$\bigcup \limits_{i=1}^{n}A_{i}$是可数集    
\end{la}
\noindent 证明:
\par \ding{172} 对$\forall i \in \mathbb{Z^{+}}$,$A_{i}$是有限集,则$\bigcup \limits_{i=1}^{n}A_{i}$是有限集显然
\par \ding{173} 若$\exists i_{0} \in \mathbb{Z}^{+}$,$A_{i_{0}}$是可数集,则$\bigcup \limits_{i=1}^{n}A_{i} = A_{i_{0}}\cup A_{1}\cup A_{2}\cup \dots\cup A_{i_{0}-1}\cup A_{i_{0}+1}\cup \dots\cup A_{n}$
\par \quad 由Th3可知:$A_{i_{0}}\cup A_{1}$是可数集,$A_{i_{0}}\cup A_{1}\cup A_{2}$是可数集
\par \quad 依此类推$\bigcup \limits_{i=1}^{n}A_{i}$是可数集

\begin{wa}
\pmb{注3:}设$A_{i}\left(i =1,2,3,\dots\right) $是互不相交的非空有限集,则$\bigcup \limits_{i=1}^{\infty}A_{i}$是可数集    
\end{wa}
\noindent 证明:
\par 记:
\begin{align*}
    A_{1} & =\left\{a_{11},a_{12},\dots,a_{1N_{1}}\right\} \\
    A_{2} & =\left\{a_{21},a_{22},\dots,a_{2N_{2}}\right\} \\
    & \vdots \\
    A_{n} & =\left\{a_{n1},a_{n2},\dots,a_{nN_{n}}\right\} \\
    & \vdots 
\end{align*}
\par 则:
$$\bigcup \limits_{i=1}^{\infty}A_{i} = \left\{a_{11},a_{12},\dots,a_{1N_{1}},a_{21},a_{22},\dots,a_{2N_{2}},\dots,a_{n1},a_{n2},\dots,a_{nN_{n}},\dots\right\}$$
\par 故:$\bigcup \limits_{i=1}^{\infty}A_{i}$是可数集,得证

\begin{td}
\pmb{Theorem \ 4:}设$A_{i}\left(i =1,2,3,\dots\right) $是可数集,则$\bigcup \limits_{i=1}^{\infty}A_{i}$是可数集    
\end{td}
\noindent 证明:
\par \ding{172}若$A_{1},\dots,A_{n},\dots$互不相交
\par \quad 已知对$\forall i \in \mathbb{Z}^{+}$,$A_{i}$是可数集
\par \quad $A_{i}$可记为:
\begin{align*}
    A_{1} & =\left\{a_{11},a_{12},a_{13},a_{14},\dots\right\} \\
    A_{2} & =\left\{a_{21},a_{22},a_{23},a_{24},\dots\right\} \\
    A_{3} & =\left\{a_{31},a_{32},a_{33},a_{34},\dots\right\} \\
    A_{4} & =\left\{a_{41},a_{42},a_{43},a_{44},\dots\right\} \\
    & \vdots 
\end{align*}
\par \quad 按对角线顺序将$\bigcup \limits_{i=1}^{\infty}A_{i}$排成无穷序列:
$$a_{11},a_{12},a_{21},a_{31},a_{22},a_{13},a_{14},a_{23},a_{32},a_{41},\dots$$
\par \quad 故$\bigcup \limits_{i=1}^{\infty}A_{i}$是可数集
\par \ding{173}若$A_{1},\dots,A_{n},\dots$有相交部分
\par \quad 令\footnote{将相交的集合重新构造的常用方法,构造后的集合的并集与原来的集合的并集相等}:
\begin{align*}
    A^{*} & = A_{1} \\
    A^{*}_{2} & = A_{2} \setminus A_{1} \\
    \vdots \\
    A^{*}_{i} & = A_{i} \setminus \left(\bigcup \limits_{j=1}^{i-1}A_{j}\right) \\
    \vdots 
\end{align*}
\par \quad \romannumeral1. 当对$\forall i \neq j$时,有$A^{*}_{i} \cap A^{*}_{j} = \varnothing $,下证\romannumeral1.成立:
\par \quad \quad 不妨设 $i > j$,则$i-1 \geqslant j$
$$A^{*}_{i} \cap A^{*}_{j} = \left[A_{i} \setminus \left(\bigcup \limits_{k=1}^{i-1}A_{k}\right)\right] \cap \left[A_{j} \setminus \left(\bigcup \limits_{l=1}^{j-1}A_{l}\right)\right] \subset \left(A_{i} \setminus A_{j}\right) \cap A_{j} = \varnothing$$
\par \quad \romannumeral2. 对$\forall n \in \mathbb{N}$,有$\bigcup \limits_{i=1}^{n}A_{i} = \bigcup \limits_{i=1}^{n}A_{i}^{*}$,下证\romannumeral2.成立:
\par \quad \quad 数学归纳法证明:
\par \quad \quad 当$n = 1$时,$A_{1} = A_{1}^{*}$成立
\par \quad \quad 假设当$n = n - 1$时,$\bigcup \limits_{i=1}^{n-1}A_{i} = \bigcup \limits_{i=1}^{n-1}A_{i}^{*}$成立
\par \quad \quad 下证当$n = n$时,$\bigcup \limits_{i=1}^{n}A_{i} = \bigcup \limits_{i=1}^{n}A_{i}^{*}$成立
\par \quad \quad 则:
\begin{align*}
    \bigcup \limits_{i=1}^{n}A_{i}^{*} & = \left(\bigcup \limits_{i=1}^{n-1}A_{i}^{*} \right) \cup A_{n}^{*} \\
    & = \left(\bigcup \limits_{i=1}^{n-1}A_{i} \right) \cup \left[A_{n} \cap \left(\bigcup \limits_{i=1}^{n-1}A_{i} \right)^{c} \ \right] \\
    & = \left[\left(\bigcup \limits_{i=1}^{n-1}A_{i} \right) \cup A_{n}\right] \cap \left[\left(\bigcup \limits_{i=1}^{n-1}A_{i} \right) \cup \left(\bigcup \limits_{i=1}^{n-1}A_{i}\right)^{c} \ \right] \\
    & = \bigcup \limits_{i=1}^{n}A_{i}
\end{align*}
\par \quad \romannumeral3.当$n \to \infty$,有$\bigcup \limits_{i=1}^{\infty}A_{i} = \bigcup \limits_{i=1}^{\infty}A_{i}^{*}$, 下证\romannumeral3.成立:
\begin{align*}
    \bigcup \limits_{i=1}^{\infty}A_{i} = \bigcup \limits_{n=1}^{\infty}\bigcup \limits_{i=1}^{n}A_{i} = \bigcup \limits_{n=1}^{\infty}\bigcup \limits_{i=1}^{n}A_{i}^{*} = \bigcup \limits_{i=1}^{\infty}A_{i}^{*}
\end{align*}
\par \quad \quad 对$\forall i \in \mathbb{N}$,有$A_{i}^{*} \subset A_{i}$
\par \quad \quad 对$\forall i \in \mathbb{N}$,$A_{i}^{*}$是有限集或可数集
\par \quad \quad 故:
\begin{align*}
    \bigcup \limits_{i=1}^{\infty}A_{i} = \bigcup \limits_{i=1}^{\infty}A_{i}^{*} = \underbrace{\bigcup \limits_{i=1}^{\infty}A_{i}^{*}} _\text{$A_{i}^{*} \neq \varnothing$} = \uline{\underbrace{\left(\bigcup A_{i}^{*}\right)}_\text{$A_{i}^{*} \neq \varnothing$且是有限集}}_{B_{1}} \cup \uline{\underbrace{\left(\bigcup A_{i}^{*}\right)}_\text{$A_{i}^{*}$是可数集}}_{B_{2}}
\end{align*}
\par \quad \quad 若非空有限集的$A_{i}^{*}$只有有限个,则$B_{1}$是有限集;
\par \quad \quad 若非空有限集的$A_{i}^{*}$有无限个(可数个),则由注3,$B_{1}$是可数集
\par \quad \quad 若可数集的$A_{i}^{*}$只有有限个,则由Th3推论知$B_{2}$是可数集;
\par \quad \quad 若可数集的$A_{i}^{*}$有无限个(可数个),则由\ding{172},$B_{2}$是可数集
\par 综上,由Th3可知,$\bigcup \limits_{i=1}^{\infty}A_{i} = B_{1} \cup B_{2}$是可数集

\begin{wa}
\pmb{注4:}1.当$A_{i}$均为可数集时,定理3的推论可简记为
$$n \cdot a = \underbrace{a + a + \dots + a }_\text{n个}= a$$
\quad \quad \quad 2.定理4的结论可简记为
$$a \cdot a = \underbrace{a + a + \dots + a }_\text{可数个}= a$$    
\end{wa}

\begin{td}
\pmb{Theorem \ 5:}有理数的全体是可数集   
\end{td}
\noindent 证明:
\par 令$A_{i} = \left\{\dfrac{1}{i} \ ,\dfrac{2}{i} \ ,\dfrac{3}{i} \ ,\dots\right\} ,i = 1,2,3,\dots$
\par 故对$\forall i \in \mathbb{Z}^{+}$,$A_{i}$是可数集,由Th4可知,$\mathbb{Q}^{+} = \bigcup \limits_{i=1}^{\infty}A_{i}$是可数集
\par 令$\varphi : \mathbb{Q}^{+} \to \mathbb{Q}^{-} , x \mapsto -x$,易知$\varphi $是双射,故$\mathbb{Q}^{+} \thicksim \mathbb{Q}^{-}$,故$\mathbb{Q}^{-}$是可数集
\par 由Th3推论可知,$\mathbb{Q} = \mathbb{Q}^{+} \cup \mathbb{Q}^{-} \cup \left\{0\right\} $是可数集

\begin{wa}
\pmb{注5:}虽然有理数在实数中稠密,但有理数集只与稀疏分布的正整数集一一对应    
\end{wa}

\begin{eg}
\pmb{例1:}设集合$A$中的元素都是直线上的开区间,满足:若开区间$K$,$J \in A$,$K \neq J$,则$K\cap J = \varnothing$
\par 求证:$A$是可数集或有限集    
\end{eg}
\noindent 证明:
\par 对$\forall K \in A$,由实数集的稠密性可知,$\exists r \in \mathbb{Q} $,$ s.t. \ r \in K$
\par 在每个$K$中取且只取一个有理数,记为$r_{K}$,令映射$\varphi : A \to \mathbb{Q} , K \mapsto \varphi (K) = r_{K}$
\par 下证:$\varphi $是单射
\par \quad \quad \quad 对$\forall K,T \in A \land K \neq J$
\par \quad \quad \quad 因$K\cap J = \varnothing$,$r_{K} \in K$,$r_{J} \in J$,故$r_{K} \neq r_{J}$,即$\varphi (K) \neq \varphi (J)$
\par \quad \quad \quad 故$\varphi $是单射,由1.3注可知$\overline{\overline{A}} \leqslant \overline{\overline{\mathbb{Q}}}$
\par \quad \quad \quad 因$\mathbb{Q}$是可数集,$A$是可数集或有限集

\begin{td}
\pmb{Theorem \ 6:}设$A_{i}\left(i =1,2,\dots,n\right) $是可数集,则$A = A_{1} \times A_{2} \times \dots \times A_{n}$是可数集    
\end{td}
\noindent 证明:数学归纳法:
\par 当$n = 1$时,$A = A_{1}$是可数集
\par 假设当$n = n - 1$时成立,即$A_{1} \times A_{2} \times \dots \times A_{n-1}$是可数集
\par 已知$A_{n}$是可数集,可记:$A_{n} = \left\{x_{1},x_{2},\dots,x_{k},\dots\right\} = \bigcup \limits_{k=1}^{\infty}\left\{x_{k}\right\} $
\par 令$\widehat{A_{k}} = A_{1} \times \dots \times A_{n-1} \times \left\{x_{k}\right\} $
\par 映射$\varphi : \widehat{A_{k}} \to A_{1} \times \dots \times A_{n-1}$,$ \left(a_{1},\dots,a_{n-1},x_{k}\right) \mapsto \left(a_{1},\dots,a_{n-1}\right) $
\par 易知$\varphi $是双射,故$\widehat{A_{k}} \thicksim A_{1} \times \dots \times A_{n-1} $,故对$\forall k \in \mathbb{Z}^{+}$,$\widehat{A_{k}} $是可数集
\par 因$A = A_{1} \times \dots \times A_{n} = \bigcup \limits_{k=1}^{\infty}\left(A_{1} \times \dots \times A_{n-1} \times \left\{x_{k}\right\} \right) = \bigcup \limits_{k=1}^{\infty}\widehat{A_{k}}$
\par 由Th4可知,$A$是可数集

\begin{eg}
\pmb{例2:}平面上坐标为有理数的点的全体构成的集合为可数集    
\end{eg}
\noindent 证明:$A = \left\{\left(x,y\right) : x \in \mathbb{Q} , y \in \mathbb{Q}\right\} = \mathbb{Q} \times \mathbb{Q}$

\begin{eg}
\pmb{例3:}元素$\left(n_{1},n_{2},\dots,n_{k}\right) $是由$k$个正整数组成的,其全体是可数集    
\end{eg}
\noindent 证明:\ $\underbrace{\mathbb{Z}^{+} \times \mathbb{Z}^{+} \times \dots \times \mathbb{Z}^{+}} _\text{k个}$

\begin{eg}
\pmb{例4:}整系数多项式
$$a_{0}x^{n} + a_{1}x^{n-1} + \dots + a_{n-1}x + a_{n}$$
\par \quad 的全体$A$是可数集    
\end{eg}
\noindent 证明:
\par 对$\forall n \in \mathbb{N}$,记$A_{n} = \left\{a_{0}x^{n} + a_{1}x^{n-1} + \dots + a_{n-1}x + a_{n} : a_{0} \in \mathbb{Z} \setminus \left\{0\right\}  , \  a_{1},\dots,a_{n} \in \mathbb{Z}\right\} $
\par 令映射$\varphi : A_{n} \to \mathbb{Z} \setminus \left\{0\right\} \times \underbrace{\mathbb{Z} \times \dots \times \mathbb{Z}}_\text{n个}$, \ $a_{0}x^{n} + a_{1}x^{n-1} + \dots + a_{n-1}x + a_{n} \mapsto \left(a_{0},a_{1},\dots,a_{n}\right) $
\par 易知$\varphi$是双射
\par 故对$\forall n \in \mathbb{N}$,$A_{n}$是可数集
\par 记$A_{0} = \mathbb{Z}$,则$A = \bigcup \limits_{n=0}^{\infty}A_{n}$是可数集

\begin{wa}
\pmb{注6:}可数个有限集的并集是有限集或可数集 \ \textcolor{red}{$\bigstar$}    
\end{wa}
\noindent 证明:
\par 设$\left\{A_{i}\right\} _{i=1}^{\infty}$是可数个有限集,下证$\bigcup \limits_{i=1}^{\infty}A_{i}$是有限集或可数集
\par 令$A_{1}^{*}$,则$A_{i}^{*} = A_{i} \setminus \left(\bigcup \limits_{j=1}^{i-1}A_{j}\right) $
\par 对$\forall i \geqslant 2$,$\left\{A_{i}^{*}\right\}$互不相交且$\bigcup \limits_{i=1}^{\infty}A_{i} = \bigcup \limits_{i=1}^{\infty}A_{i}^{*}$
\par 因对$\forall i \in \mathbb{N}$,有$A_{i}^{*} \subset A_{i}$,故对$\forall i \in \mathbb{N}$,$A_{i}^{*}$是有限集
\par \ding{172}若只有有限个$A_{i}^{*}$不为$\varnothing$,则
\par \quad $\bigcup \limits_{i=1}^{\infty}A_{i} = \bigcup \limits_{i=1}^{\infty}A_{i}^{*}(A_{i}^{*} \neq \varnothing)$是有限集
\par \ding{173}若有无限个(可数个)$A_{i}^{*}$不为$\varnothing$,则
\par \quad $\bigcup \limits_{i=1}^{\infty}A_{i} = \bigcup \limits_{i=1}^{\infty}A_{i}^{*}(A_{i}^{*} \neq \varnothing)$是可数个互不相交的非空有限集的并,
\par \quad  由注3可知,$\bigcup \limits_{i=1}^{\infty}A_{i}$是可数集

\begin{td}
\pmb{Theorem \ 6:}代数数(整系数多项式的根)的全体$B$是可数集    
\end{td}
\noindent 证明:
\par 令$A = $ \ \{整系数多项式\}  
\par 已知$A$是可数集,可记$A = \left\{P_{1},P_{2},\dots,P_{k},\dots\right\} $
\par 对$\forall k \in \mathbb{N}$,记$B_{k} = $ \ \{$P_{k}$的根\}  
\par 则对$\forall k \in \mathbb{N}$,$B_{k}$有根
\par 故$B = \bigcup \limits_{k=1}^{\infty}B_{k}$是有限集或可数集
\par 因对$\forall n \in \mathbb{N}$,$n$是多项式$n-1$的根
\par 故对$\forall n \in \mathbb{N}$,$n \in B$
\par 故$\mathbb{N} \subset B$,$B$是可数集
\newpage
\section{不可数集合}
\pmb{本节重点:明晰不可数集合的几种常见解题思路}
\begin{td}
\pmb{定义:}不是可数集合的无限集合称为不可数集合    
\end{td}

\begin{td}
\pmb{Theorem \ 1:}实数集$\mathbb{R}$是不可数集    
\end{td}
\noindent 证明:
\par 已知$\mathbb{R} \thicksim \left(0,1\right) $,故只需证$\left(0,1\right) $是不可数集
\par $\looparrowright$ 正规表示:$\left(0,1\right) $中的任意实数$a$可唯一表示为十进位无穷小数
$$a = 0.a_{1}a_{2}\dots a_{n}\dots = \sum \limits_{n=1}^{\infty}\dfrac{a_{n}}{10^{n}} \quad \left(\forall n \in \mathbb{N},a_{n} \in \left\{0,1,\dots,9\right\} \right) $$
\par \quad \ 其中小数位不全为9也不以0为循环节, \ $\left(0,1\right) $中实数与其正规表示一一对应$\looparrowleft $
\\ 反证法:假设$\left(0,1\right) $是可数集,则其可排成无穷序列,即
$$\left(0,1\right) = \left\{a^{\left(1\right) },a^{\left(2\right) },\dots,a^{\left(n\right) },\dots\right\} $$
\par \quad \quad 对$\forall n \in \mathbb{N}$,有:
\begin{align*}
    a^{\left(1\right) } & = 0.\textcolor{cyan}{a_{1}^{\left(1\right)}}a_{2}^{\left(1\right)}a_{3}^{\left(1\right)}\dots a_{n}^{\left(1\right)} \dots\\
    a^{\left(2\right) } & = 0.a_{1}^{\left(2\right)}\textcolor{cyan}{a_{2}^{\left(2\right)}}a_{3}^{\left(2\right)}\dots a_{n}^{\left(2\right)} \dots\\
    a^{\left(3\right) } & = 0.a_{1}^{\left(3\right)}a_{2}^{\left(3\right)}\textcolor{cyan}{a_{3}^{\left(3\right)}}\dots a_{n}^{\left(3\right)} \dots\\
    \vdots \\
    a^{\left(n\right) } & = 0.a_{1}^{\left(n\right)}a_{2}^{\left(n\right)}a_{3}^{\left(n\right)}\dots \textcolor{cyan}{a_{n}^{\left(n\right)}} \dots
\end{align*}
\par \quad \quad 利用对角线上的数字构造无穷小数
$$a = 0.a_{1}a_{2}\dots a_{n} \dots \quad among: \ a_{n} = 
\begin{cases} 
    1 & ,\text{if } a_{n}^{\left(n\right)} \neq 1\\ 
    2 & ,\text{if } a_{n}^{\left(n\right)} = 1
\end{cases} $$
\par \quad \quad 则$a \in \left(0,1\right)$,对$\forall n \in \mathbb{N}$,有$a \neq a^{\left(n\right)}$
\par \quad \quad 即$\left(0,1\right) \neq \left\{a^{\left(1\right) },a^{\left(2\right) },\dots,a^{\left(n\right) },\dots\right\} $,$\left(0,1\right)$不能表示为无穷序列,矛盾
\par \quad \quad 故假设不成立,$\left(0,1\right)$是不可数集,即$\mathbb{R}$是不可数集

\begin{la}
\pmb{推论1:}若用$c$表示实数集$\mathbb{R}$的基数,$a$表示正整数集$\mathbb{Z}^{+}$的基数,则$c > a$,
\par \quad \quad 称$c$为连续基数,记为$\aleph $
%%推论1有一个口述证明,还未整理%%    
\end{la}

\begin{td}
\pmb{Theorem \ 2:}任意区间$\left(a,b\right) $,$\left[a,b\right) $,$\left(a,b\right] $,$\left[a,b\right] $,$\left(0,\infty\right) $,$\left[0,\infty\right) $,均具有连续基数$c$,
\par \quad \quad \quad \quad 其中$a < b$    
\end{td}
\noindent 证明:
\par 令映射$\varphi : \left(a,b\right) \to \left(0,1\right) $,$x \mapsto \dfrac{x - a}{b - a}$
\par 易知$\varphi$是双射,则有$\left(a,b\right) \thicksim \left(0,1\right) \thicksim \mathbb{R}$
\par 已知$\left(a,b\right) \subset \left(a,b\right] \subset \left[a,b\right] \subset \mathbb{R}$ 且$\left(a,b\right) \thicksim \mathbb{R}$
\par 由$Bernstein$定理推论可知,$\left(a,b\right) \thicksim \left(a,b\right] \thicksim \left[a,b\right] \thicksim \mathbb{R}$ 
\par 同理可得,$\left(a,b\right) \thicksim \left[a,b\right) \thicksim \mathbb{R}$ 
\par 已知$\left(0,1\right) \subset \left(0,\infty\right) \subset \left[0,\infty\right) \subset \mathbb{R}$ 且$\left(0,1\right) \thicksim \mathbb{R}$
\par 故$\left(0,1\right) \thicksim \left(0,\infty\right) \thicksim \left[0,\infty\right) \thicksim \mathbb{R}$
\par 综上:区间均具有连续基数$c$

\begin{td}
\pmb{Theorem \ 3:}设$A_{1},A_{2},\dots,A_{n},\dots$是一列\footnote{注:一列集合暗指可数个,一族集合情况未知}互不相交的集合,它们的基数都是$c$,
\par \quad \quad \quad \quad 则$\bigcup \limits_{n=1}^{\infty}A_{n}$的基数也是$c$ 
\normalsize    
\end{td}
\noindent 证明:
\par 对$\forall n \in \mathbb{N}$,令$I_{n} = \left[n-1,n\right) $,则$I_{1},I_{2},\dots,I_{n},\dots$互不相交
\par 由Th2可知,对$\forall n \in \mathbb{N}$,有$\overline{\overline{I_{n}} } = c$,故对$\forall n \in \mathbb{N}$,有$A_{n} \thicksim I_{n}$
\par 由1.3引理可知,$\bigcup \limits_{n=1}^{\infty}A_{n} \thicksim \bigcup \limits_{n=1}^{\infty}I_{n} = \left[0,\infty\right) $
\par 故$\overline{\overline{\bigcup \limits_{n=1}^{\infty}A_{n}}} = c $

\begin{td}
\pmb{Theorem \ 4:}设有一列集合$\left\{A_{n} : n \in \mathbb{Z}^{+}\right\} $,$\overline{\overline{A_{n}} } = c \ \left(n = 1,2,\dots\right) $,而$A = \prod \limits_{n=1}^{\infty}A_{n}$,
\par \quad \quad \quad \quad 则$\overline{\overline{A}} = c$    
\end{td}
\noindent 证明\footnote{证明基数值时,若找不到对等关系,可以找单射或满射,利用1.3注,建立两个不等式关系,最后变成等式}:
\par \ding{172} 已知对$\forall n \in \mathbb{Z}^{+}$,有$\overline{\overline{A_{n}} } = c$
\par \quad 故对$\forall n \in \mathbb{Z}^{+}$,有$A_{n} \thicksim \left(0,1\right) $,则$\exists \varphi_{n} : \left(0,1\right) \stackrel{1-1}{\to} A_{n}$
\par \quad 令$\varphi : \left(0,1\right) \to A$,$x \mapsto \varphi (x) = \left(\varphi _{1}(x),\varphi _{2}(x),\dots,\varphi _{n}(x),\dots\right) $,易知$\varphi$是单射
\par \quad 故由1.3注可知,$\overline{\overline{A}} \geqslant \overline{\overline{\left(0,1\right)}} = c$
\par \ding{173} 对$\forall \overline{x} \in A$,记$\overline{x} = \left(\overline{x_{1}} ,\overline{x_{2}}, \dots,\overline{x_{n}} , \dots\right)$,其中对$\forall n \in \mathbb{Z}^{+}$,有$\overline{x_{n}} \in A_{n} $
\par \quad 对$\forall n \in \mathbb{Z}^{+}$,令$x_{n} = \varphi^{-1}_{n}\left(\overline{x_{n}}\right) \in \left(0,1\right) $ 
\par \quad 对$\forall n \in \mathbb{Z}^{+}$,有:
\begin{align*}
    x_{1} & = 0.x_{11}x_{12}x_{13}\dots \\
    x_{2} & = 0.x_{21}x_{22}x_{23}\dots \\
    x_{3} & = 0.x_{31}x_{32}x_{33}\dots \\
    \vdots
\end{align*}
\par \quad 令$\psi : A \to \left(0,1\right) $,$\overline{x} \mapsto 0.x_{11}x_{12}x_{21}x_{31}x_{22}x_{13} \dots$,易知$\psi$是单射
\par \quad $\looparrowright$ 若$\overline{x} , \overline{y} \in A $,且$\overline{x} = \overline{y}$,则$\exists n \in \mathbb{Z}^{+}$,有$\overline{x_{n}} \neq \overline{y_{n}}$,故$x_{n} \neq y_{n}$
\par \quad \quad 故$\exists m \in \mathbb{Z}^{+}$,$s.t. \ x_{nm} \neq y_{nm}$,即$\psi \left(\overline{x} \right) \neq \psi \left(\overline{y} \right)$ \quad \quad \quad \quad \quad$\looparrowleft $
\par \quad 故由1.3注可知,$\overline{\overline{A}} \leqslant \overline{\overline{\left(0,1\right)}} = c$
\par 综上:由$Bernstein$定理,$\overline{\overline{A}} = c$

\begin{td}
\pmb{定义 \ }设$n$是正整数,由$n$个实数$x_{1},x_{2},\dots,x_{n} $按确定的次序排成的数组$\left(x_{1},x_{2},\dots,x_{n}\right) $的全体称为\uline{$n$维欧几里得空间}(简称\uline{欧式空间}),记为$\mathbb{R}^{n}$
\par \quad 每个组$\left(x_{1},x_{2},\dots,x_{n}\right) $称为\uline{欧几里得空间的点},又称$x_{i}$为点$\left(x_{1},x_{2},\dots,x_{n}\right) $的第$i$个\uline{坐标},
记:
$$\mathbb{R}^{\infty} = \left\{\left(x_{1},x_{2},\dots,x_{n}\right) : x_{i} \in \mathbb{R} , i = 1,2,\dots\right\} $$    
\end{td}

\begin{la}
\pmb{推论:}$\mathbb{R}^{\infty}$的基数为$c$    
\end{la}
\noindent 证明: $\underbrace{\mathbb{R} \times \mathbb{R} \times \dots}_\text{可数个}$

\begin{td}
\pmb{Theorem \ 5:}$n$维欧几里得空间$\mathbb{R}^{n}$的基数为$c$    
\end{td}
\noindent 证明:
\par \ding{172}令$\varphi _{1} : \mathbb{R}^{n} \to \mathbb{R}^{\infty}$,$x = \left(x_{1},\dots,x_{n} \right) \mapsto \left(x_{1},\dots,x_{n},0,0,\dots\right)  $
\par \quad 易知$\varphi_{1} $是单射,即$\overline{\overline{\mathbb{R}^{n}}} \leqslant \overline{\overline{\mathbb{R}^{\infty}}} = c$
\par \ding{173}令$\varphi _{2} : \mathbb{R} \to \mathbb{R}^{n}$,$x \mapsto \left(x,0,\dots,0\right)  $
\par \quad 易知$\varphi_{2} $是单射,即$\overline{\overline{\mathbb{R}^{n}}} \geqslant  \overline{\overline{\mathbb{R}}} = c$
\par 由$Bernstein$定理,$\overline{\overline{\mathbb{R}^{n}}} = c$

\begin{la}
\pmb{推论2:}设有一列集合$B_{n} : n \in \mathbb{Z}^{+}$,$B_{n} = \left\{0,1\right\} \left(n = 1,2,\dots\right) $,而$B = \prod \limits_{n=1}^{\infty}B_{n}$,则$\overline{\overline{B}} = c$    
\end{la}
\noindent 证明:
\par \ding{172} 对$\forall n \in \mathbb{Z}^{+}$,令$A_{n} = \left(0,1\right) $,令$f : \left\{0,1\right\} \to \left(0,1\right) ,
\begin{cases}
    0 &\mapsto \ 0.1  \\
    1 &\mapsto \ 0.2
\end{cases} $
\par 易知$f$是单射
\par 令$\varphi :B = \prod \limits_{n=1}^{\infty}B_{n} \to \prod \limits_{n=1}^{\infty}A_{n}$,$\left( b_{1},b_{2},\dots,b_{n},\dots \right)\mapsto \left(f(b_{1}),f(b_{2}),\dots,f(b_{n}),\dots\right) $
\par 易知$\varphi$是单射
\par 故$\overline{\overline{B}} \leqslant \overline{\overline{\prod \limits_{n=1}^{\infty}A_{n}}} \stackrel{Th4}{=} c$
\par \ding{173} 对$\forall x \in \left(0,1\right) $,将$x$用二进位表示
\par \quad $x = 0.a_{1}a_{2}\dots a_{n}$,其中对$\forall n \in \mathbb{Z}^{+}$,有$a_{n} \in \left\{0,1\right\} $,
\par \quad $x$的小数位不全为1也不以0为循环节
\par \quad 令$\psi : \left(0,1\right) \to B$,$x \mapsto \left(a_{1},a_{2},\dots,a_{n},\dots\right) $
\par \quad $\looparrowright $对$\forall x , y \in \left(0,1\right) $且$x \neq y$,记$x = 0.a_{1}a_{2}\dots a_{n}\dots$,$y = b_{1}b_{2}\dots b_{n}\dots$
\par \quad \quad \ 则$\exists n om \mathbb{Z}^{+}$,有$a_{n} \neq b_{n}$,故$\psi (x) \neq \psi (y)$,故$\psi $是单射
\par \quad 故$\overline{\overline{B}} \geqslant \overline{\overline{\left(0,1\right) }} = c$
\par 综上:由$Bernstein$定理,$\overline{\overline{B}} = c$

\begin{wa}
\pmb{注:}定理4,定理5和推论2分别可简记为
$$ c^{a} = c ,\quad c^{n} = c , \quad 2^{a} = c$$
\end{wa}
\par Th4: $$\overline{\overline{\prod \limits_{n=1}^{\infty}A_{n}}} = c \quad \underbrace{c \cdot c\cdot c\cdot \dots} _\text{a个} = c^{a} = c$$
\par Th5: $$\overline{\overline{\mathbb{R}^{n}}} = c \quad \underbrace{c \cdot c\cdot \dots \cdot c} _\text{n个} = c^{n} = c$$
\par 推论2:$$\overline{\overline{\prod \limits_{n=1}^{\infty}B_{n}}} = c \quad \underbrace{2 \cdot 2\cdot 2\cdot \dots} _\text{a个} = 2^{a} = c$$

\begin{td}
    \pmb{Theorem \ 6:}设$M$是任意集合,其所有子集作成新的集合$\mu $,则$\overline{\overline{\mu }} > \overline{\overline{M}} $
\end{td}
\noindent 证明:
\par 对$\mu = \left\{E : E \subset M\right\} $,令映射$\varphi : N \to \mu  ,x \mapsto \left\{x\right\} $,易知$\mu$是单射,即有$\overline{\overline{M}} \leqslant \overline{\overline{\mu }}$
\par 下证:$\overline{\overline{M}} \neq \overline{\overline{\mu }}$
\par 反证法:假设$\overline{\overline{M}} = \overline{\overline{\mu }}$,则$M \thicksim \mu$,故$\exists \psi : M \stackrel{1-1}{\to} \mu$
\par \quad \quad \quad \quad 故对$\forall \alpha \in M$,令$M_{\alpha } = \psi (\alpha ) \in \mu$,则有$M_{\alpha } \subset M$
\par \quad \quad \quad \quad 令$M' = \left\{\alpha \in M : \alpha \notin M_{\alpha }\right\} $,则$M' \subset M$,故$M' \in \mu = \psi (M)$
\par \quad \quad \quad \quad 故$\exists \alpha ' \in M$,$s.t. \ M' = \psi (\alpha ')$
\par \quad \quad \quad \quad \ding{172} 若$\alpha ' \in M'$,则由$M'$的定义,$\alpha ' \notin \psi (\alpha ') = M'$,矛盾
\par \quad \quad \quad \quad \ding{173} 若$\alpha ' \notin M'$,则$\alpha ' \notin \psi (\alpha ')$,由$M'$的定义,$\alpha ' \in M'$,矛盾
\par \quad \quad \quad \quad 故假设不成立,故$\overline{\overline{M}} \neq \overline{\overline{\mu }}$,即$\overline{\overline{M}} < \overline{\overline{\mu }}$

\begin{wa}
    \pmb{注:}集合$M$的所有子集组成的集合(幂集)记为$2^{M}$,则$\overline{\overline{M}} <\overline{\overline{2^{M}}} $
    \par 没有最大的基数,无限集合的不同基数有无限之多
\end{wa}

\begin{eg}
    \pmb{例1:}若$f(x)$是定义在$E$上的函数,则
    $$\left\{x : \left\lvert f(x)\right\rvert = 0\right\} = \bigcap \limits_{\varepsilon \in \mathbb{R}^{+}} \left\{x : \left\lvert f(x)\right\rvert < \varepsilon \right\} = \bigcap \limits_{n=1}^{\infty} \left\{x : \left\lvert f(x)\right\rvert < \dfrac{1}{n} \right\} $$
\end{eg}
\noindent 证明:$\xrightarrow{regard \ as} \ A = \bigcap \limits_{\varepsilon \in \mathbb{R}^{+}}A_{\varepsilon } = \bigcap \limits_{n=1}^{\infty} B_{n}$
\par 显然:$A \subset \bigcap \limits_{\varepsilon \in \mathbb{R}^{+}}A_{\varepsilon } \subset \bigcap \limits_{n=1}^{\infty} B_{n}$
\par 下证:$\bigcap \limits_{n=1}^{\infty} B_{n} \subset A$
\par 对$\forall x \in \bigcap \limits_{n=1}^{\infty} B_{n} $,有对$\forall n \in \mathbb{R}$,$x \in B_{n}$,故对$\forall n \in \mathbb{R}$,有$\left\lvert f(x)\right\rvert < \dfrac{1}{n} $
\par 令$n \to \infty$,得$\left\lvert f(x)\right\rvert \leqslant 0$,即$\left\lvert f(x)\right\rvert = 0$,故$x \in A$
\par 故$\bigcap \limits_{n=1}^{\infty} B_{n} \subset A$
\par 综上:$ A = \bigcap \limits_{\varepsilon \in \mathbb{R}^{+}}A_{\varepsilon } = \bigcap \limits_{n=1}^{\infty} B_{n}$

\begin{eg}
    \pmb{例2:}若$\left\{f_{n}(x)\right\} $是定义在$E$上的函数列,则 
    \begin{align*}
        (1)\left\{x : \lim \limits_{n \to \infty} f_{n}(x) = 0\right\} & = \bigcap \limits_{\varepsilon \in \mathbb{R}^{+}} \bigcup \limits_{N=1}^{\infty} \bigcap \limits_{n=N}^{\infty} \left\{x : \left\lvert f_{n}(x)\right\rvert < \varepsilon \right\} \\
        & = \bigcap \limits_{k=1}^{\infty} \bigcup \limits_{N=1}^{\infty} \bigcap \limits_{n=N}^{\infty}\left\{x : \left\lvert f_{n}(x)\right\rvert < \dfrac{1}{k} \right\}
    \end{align*}
\end{eg}
\noindent 证明:$\xrightarrow{regard \ as} \ (1) \xlongequal{1.2 \ eg12} A = B$
\par 显然:$A \subset B$,下证:$B \subset A$
\par 对$\forall x \in B$,则对$\forall k \in \mathbb{N}$,$\exists N \in \mathbb{N}$,对$\forall N \in \mathbb{N}$,有$\left\lvert f_{n}(x)\right\rvert < \dfrac{1}{k} \dots \dots \dots \dots $ \ding{172}
\par 对$\forall \varepsilon > 0$,令$k = \left[\dfrac{1}{\varepsilon}\right] +1 \in \mathbb{N}$,则$k > \dfrac{1}{\varepsilon }$,即$\dfrac{1}{k} < \varepsilon $
\par 由\ding{172}式可知,$\exists N \in \mathbb{N}$,对$\forall n \geqslant N$,有$\left\lvert f_{n}(x)\right\rvert < \dfrac{1}{k} <\varepsilon $
\par 故$x \in \bigcap \limits_{\varepsilon \in \mathbb{R}^{+}} \bigcup \limits_{N=1}^{\infty} \bigcap \limits_{n=N}^{\infty} \left\{x : \left\lvert f_{n}(x)\right\rvert < \varepsilon \right\} = A$
\par 故$B \subset A$
\par 综上:$(1) = A = B$

\begin{eg}
    \begin{align*}
        (2)\left\{x : \lim \limits_{n \to \infty} f_{n}(x) \neq 0 \ or \nexists \right\} & = \bigcup \limits_{\varepsilon \in \mathbb{R}^{+}} \bigcap \limits_{N=1}^{\infty} \bigcup \limits_{n=N}^{\infty} \left\{x : \left\lvert f_{n}(x)\right\rvert \geqslant \varepsilon \right\} \\
        & = \bigcup \limits_{k=1}^{\infty} \bigcap \limits_{N=1}^{\infty} \bigcup \limits_{n=N}^{\infty}\left\{x : \left\lvert f_{n}(x)\right\rvert \geqslant \dfrac{1}{k} \right\}
    \end{align*}
\end{eg}
\noindent 证明:$\xrightarrow{regard \ as} \ (2) \xlongequal{1.2 \ eg12} A = B$
\par 显然:$A \supset B$ ,下证:$A \subset B$
\par 对$\forall x \in A$,$\exists \varepsilon >0$,对$\forall N \in \mathbb{N}$,$\exists n \geqslant N$,有$\left\lvert f_{n}(x)\right\rvert \geqslant \varepsilon \dots \dots \dots \dots $ \ding{173}
\par 令$k = \left[\dfrac{1}{\varepsilon}\right] +1 \in \mathbb{N}$,则$k > \dfrac{1}{\varepsilon }$,即$\dfrac{1}{k} < \varepsilon $
\par 由\ding{173}式可知,对$\forall N \in \mathbb{N}$,$\exists n \geqslant N$,有$\left\lvert f_{n}(x)\right\rvert \geqslant \varepsilon > \dfrac{1}{k}$
\par 故$x \in \bigcup \limits_{k=1}^{\infty} \bigcap \limits_{N=1}^{\infty} \bigcup \limits_{n=N}^{\infty}\left\{x : \left\lvert f_{n}(x)\right\rvert \geqslant \dfrac{1}{k} \right\} = B$
\par 故$A \subset B$
\par 综上:$(2) = A = B$

\chapter{第二章 \ 点集}
\paragraph{
本章主要介绍集合内各元素之间的关系,研究能够度量元素间距离的集合,重点考察度量空间这一概念
}

\newpage 
\section{度量空间,n维欧式空间}
\input{Chapter2/2.1.tex}
\newpage
\section{聚点,内点,界点}
\input{Chapter2/2.2.tex}
\newpage
\section{开集,闭集,完备集}
\section{直线上的开集、闭集及完备集的构造}
\section{Cantor三分集}

\chapter{第三章 \ 测度论} 
\section{外测度}
\section{可测集}
\section{可测集类}
\section{不可测集*}

\chapter{第四章 \ 可测函数} 
\section{可测函数及其性质}
\section{Eropob定理}%Егоров 俄语宏包没下载
\section{可测函数的构造}
\section{依测度收敛}

\chapter{第五章 \ 积分论} 
\section{Riemann积分的局限性,Lebesgue积分简介}
\section{非负简单函数的Lebesgue积分}
\section{非负可测函数的Lebesgue积分}
\section{一般可测函数的Lebesgue积分}
\section{Riemann积分和Lebesgue积分}
\section{Lebesgue积分的几何意义,Fubini定理}

\chapter{附录}


\end{document}