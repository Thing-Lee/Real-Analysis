\pmb{本节重点:理解对等本质}
\\ \hspace*{\fill}\\
\pmb{例2:}正奇数全体与正偶数全体对等
\[\varphi :x\mapsto x+1\]
\pmb{例3:}正整数全体与正偶数全体对等
\[\varphi :x\mapsto 2x\]
\pmb{例4:}$\left(0,1\right) $与$\mathbb{R}$对等
\[\varphi :x\mapsto \tan \left(\pi x - \dfrac{\pi}{2}\right)\]
\pmb{注:}例3与例4说明无限集可以与其真子集对等

\begin{td}
\pmb{Theorem \ 1:}对任意集合$A$,$B$,$C$,均有:
\par (1)自反性:$A\thicksim A$
\par (2)对称性:$A\thicksim B \Rightarrow B\thicksim A$
\par (3)传递性:$A\thicksim B,B\thicksim C\Rightarrow A\thicksim C$    
\end{td}
\noindent 证明:
\par (1) $\varphi :x\mapsto x$
\par (2)已知$A\thicksim B$,故$\exists$双射$\varphi$,$s.t. \ A \to B$,所以有$\varphi ^{-1}$,$s.t. \ B \to A$,即$B\thicksim A$
\par (3)已知$A\thicksim B$且$B\thicksim C$
\par \quad \ 故$\exists$双射$\varphi_{1}$,$s.t. \ A \to B$且$\exists$双射$\varphi_{2}$,$s.t. \ B \to C$
\par \quad \ 令$\varphi :A \to B \to C$,则$\varphi = \varphi _{1}\circ \varphi _{2}$
\par \quad \ 易知$\varphi $是双射,故$A\thicksim C$

\begin{wa}
\pmb{\textcolor{red}{$\bigstar$注:}}
\par (1)若$A \subset B$,则$\overline{\overline{A}} \leqslant \overline{\overline{B}}$
\par (2)设$A$,$B$是两个集合,若存在$A$到$B$的单射,则$\overline{\overline{A}} \leqslant \overline{\overline{B}}$
\par (3)设$A$,$B$是两个集合,若存在$A$到$B$的满射,则$\overline{\overline{A}} \geqslant \overline{\overline{B}}$    
\end{wa}
\noindent 证明:
\par (1) (\romannumeral1)若$A\thicksim B$,则$\overline{\overline{A}} = \overline{\overline{B}}$
\par \quad \ (\romannumeral2)若$A\nsim B$,则$A \subsetneq B$,而$A \thicksim A\subsetneq B$,故$\overline{\overline{A}} < \overline{\overline{B}}$
\par (2) $\exists$单射$\varphi ,\ s.t. \ A \to B$
\par \quad \ 若$\varphi :A \to \varphi (A)$是双射
\par \quad \ 则$A \thicksim \varphi (A)$,$\overline{\overline{A}} = \overline{\overline{\varphi (A)}}$
\par \quad \ 易知$\varphi (A)\subset B$,故$\overline{\overline{\varphi (A)}} \leqslant \overline{\overline{B}}$
\par \quad \ 故$\overline{\overline{A}} \leqslant \overline{\overline{B}}$
\par (3) $\exists$满射$\varphi ,\ s.t. \ A \to B$
\par \quad \ 即对$\forall y \in B$,$\exists x \in A$,$s.t. \ y = \varphi (x)$
\par \quad \ 在$\left\{x \in A : y = \varphi (x)\right\} $中取且只取一个元素,记为$x_{y}$,令$A_{0} = \left\{x_{y} : y \in B\right\} \subset A$
\par \quad \ 易知$\varphi$是$A_{0} \to B$双射,故有$A_{0}\thicksim B$
\par \quad \ 即$\overline{\overline{B}} = \overline{\overline{A_{0}}} \leqslant \overline{\overline{A}}$

\begin{td}
\pmb{Definition \ 7:}设$A$,$B$是两个集合,如果$A$不与$B$的对等,但存在$B$的真子集$B^{*}$,有$A \thicksim B^{*}$,则称$A$比$B$有较小的基数,并记为$\overline{\overline{A}} \leqslant \overline{\overline{B}}$    
\end{td}

\begin{la}
\pmb{引理:}设$\left\{A_{\lambda } : \lambda \in \varLambda \right\}$ ,$\left\{B_{\lambda } : \lambda \in \varLambda \right\}$是两个集族,$\varLambda $是一个指标集,又$\forall \lambda  \in \varLambda $,$A_{\lambda } \thicksim B_{\lambda}$,而且$\left\{A_{\lambda } : \lambda \in \varLambda \right\}$ 中的集合两两不相交,$\left\{B_{\lambda } : \lambda \in \varLambda \right\}$中的集合两两不相交,那么
\begin{equation}
 \bigcup \limits_{\lambda \in \varLambda }A_{\lambda}\thicksim \bigcup \limits_{\lambda \in \varLambda }B_{\lambda}   \tag{le} \label{Th2}
\end{equation}
\end{la} 
\noindent 证明:
\par 因对$\forall \lambda \in \varLambda $,有$A_{\lambda } \thicksim B_{\lambda}$
\par 故对$\forall \lambda \in \varLambda $,$\exists$双射$\varphi _{\lambda } : A_{\lambda }\to B_{\lambda }$
\par 对$\forall x \in \bigcup \limits_{\lambda \in \varLambda }A_{\lambda}$,由于$\left\{A_{\lambda } \right\}$两两不相交
\par 故$\exists \lambda \in \varLambda $,$s.t. \ x \in A_{\lambda }$
\par 令$\varphi (x) = \varphi _{\lambda }(x) \in B_{\lambda }\subset \bigcup \limits_{\lambda \in \varLambda }B_{\lambda}$
\par 则有$\varphi : \bigcup \limits_{\lambda \in \varLambda }A_{\lambda}\to \bigcup \limits_{\lambda \in \varLambda }B_{\lambda}$
\par 对$\forall y \in \bigcup \limits_{\lambda \in \varLambda }B_{\lambda}$,由于$\left\{B_{\lambda }\right\} $两两不相交
\par 故$\exists \lambda \in \varLambda $,$s.t. \ y \in B_{n}$
\par 因双射$\varphi _{\lambda } : A_{\lambda }\to B_{\lambda }$
\par 故$\exists x \in A_{\lambda }\subset \bigcup \limits_{\lambda \in \varLambda }A_{\lambda}$,$s.t. \ y = \varphi _{\lambda }(x)$
\par 即$y =\varphi (x)$
\par 故$\varphi $是双射,$\bigcup \limits_{\lambda \in \varLambda }A_{\lambda}\thicksim \bigcup \limits_{\lambda \in \varLambda }B_{\lambda}$

\begin{td}
\pmb{Theorem \ 2 \ (Bernstein定理):}设$A$,$B$,是两个非空集合,若$A$对等于$B$的一个子集,$B$对等于$A$的一个子集,则$A$与$B$对等,即:
\begin{center}
    若$\overline{\overline{A}} \leqslant \overline{\overline{B}}$,$\overline{\overline{B}} \leqslant \overline{\overline{A}}$,则$\overline{\overline{A}} = \overline{\overline{B}}$ 
\end{center}    
\end{td}
\noindent 证明:
\par 已知:
$$A \thicksim B_{1}\subset  B \quad B\thicksim A_{1} \subset  A $$
\par 则:
\begin{center}
$\exists$双射$\varphi _{1} : A \to B_{1}$ ,$\exists$双射$\varphi _{2} : B \to A_{1}$ 
\end{center}
\par 令:
$$A_{2} = \varphi _{2}(B_{1})$$ 
\par 由于:
$$B_{1}\subset B$$
\par 则:
$$A_{2}\subset \varphi _{2}(B) = A_{1}$$
\par 又因:
$$\varphi _{2}: B_{1}  \stackrel{1-1}{\to} A_{2}$$
\par 故有:
$$A \stackrel{\varphi _{1}}{\thicksim }B_{1}\stackrel{\varphi _{2}}{\thicksim }A_{2}$$
\par 令:
$$B_{2} = \varphi _{1}(A_{1})$$
\par 则:
$$B_{2}\subset \varphi _{1}(A) = B_{1},\quad \varphi _{1}: A_{1}  \stackrel{1-1}{\to} B_{2}$$
\par 故有:
$$B \stackrel{\varphi _{2}}{\thicksim }A_{1}\stackrel{\varphi _{1}}{\thicksim }B_{2}$$
\par 令:
$$A_{3} = \varphi _{2}(B_{2}) \quad B_{3} = \varphi _{1}(A_{2})$$
$$A_{4} = \varphi _{2}(B_{3}) \quad B_{4} = \varphi _{1}(A_{3})$$
\par 则:
$$A\supset A_{1}\supset A_{2}\supset A_{3}\supset \dots$$
$$B\supset B_{1}\supset B_{2}\supset B_{3}\supset \dots$$
\par 且:
$$A \stackrel{\varphi _{1}}{\thicksim }B_{1}\stackrel{\varphi _{2}}{\thicksim }A_{2}\stackrel{\varphi _{1}}{\thicksim }B_{3}\stackrel{\varphi _{2}}{\thicksim }A_{4}\thicksim \dots \stackrel{\varphi _{1}}{\thicksim }B_{2n-1}\stackrel{\varphi _{2}}{\thicksim }A_{2n} \thicksim \dots$$
$$B\stackrel{\varphi _{2}}{\thicksim }A_{1}\stackrel{\varphi _{1}}{\thicksim }B_{2}\stackrel{\varphi _{2}}{\thicksim }A_{3}\stackrel{\varphi _{1}}{\thicksim }B_{4}\thicksim \dots \stackrel{\varphi _{2}}{\thicksim }A_{2n-1}\stackrel{\varphi _{1}}{\thicksim }B_{2n} \thicksim \dots $$
\par 令$\varphi =\varphi _{1}\circ \varphi _{2}$,则:
$$A \stackrel{\varphi}{\thicksim }A_{2}\stackrel{\varphi}{\thicksim }A_{4}\stackrel{\varphi}{\thicksim } \dots \stackrel{\varphi}{\thicksim}A_{2n} \thicksim \dots$$
$$A_{1} \stackrel{\varphi}{\thicksim }A_{3}\stackrel{\varphi}{\thicksim }A_{5}\stackrel{\varphi}{\thicksim } \dots \stackrel{\varphi}{\thicksim}A_{2n-1} \thicksim \dots$$
\par 故对$\forall n \in \mathbb{Z}^{+}$,$\varphi $是双射:
$$A_{2} = \varphi (A),\dots,A_{n+2} = \varphi (A_{n})$$
\par 故对$\forall n \in \mathbb{Z}^{+}$:
$$A_{2}\thicksim A,\dots,A_{n+2}\thicksim A_{n}$$
\par 故对$\forall n \in \mathbb{Z}^{+}$,有:
\begin{equation}
   A_{n}\setminus A_{n+1}\thicksim A_{n+2}\setminus A_{n+3} \tag{*} \label{6}
\end{equation}
\par 将$A$与$A_{1}$分解为互不相交的子集的并:
\begin{align*}
    A & = \left(A\setminus A_{1}\right) \cup A_{1} \\
    & = \left(A\setminus A_{1}\right) \cup \left(A_{1}\setminus A_{2}\right) \cup A_{2}\\
    & = \left(A\setminus A_{1}\right) \cup \left(A_{1}\setminus A_{2}\right) \cup \left(A_{2}\setminus A_{3}\right) \cup \dots \cup D\\
    A_{1} & = \left(A_{1}\setminus A_{2}\right) \cup A_{2} \\
    & = \left(A_{1}\setminus A_{2}\right) \cup \left(A_{2}\setminus A_{3}\right) \cup A_{3}\\
    & = \left(A_{1}\setminus A_{2}\right) \cup \left(A_{2}\setminus A_{3}\right) \cup \left(A_{3}\setminus A_{4}\right) \cup \dots \cup D_{1}
\end{align*}
\par 其中:
$$D = A\cap A_{1} \cap A_{2} \cap A_{3} \cap \dots$$
$$D_{1} = A_{1} \cap A_{2} \cap A_{3} \cap A_{4} \cap \dots$$
\par 由于:
$$A\supset A_{1}$$
\par 故有:
$$D = D_{1},\quad D\thicksim D_{1}$$
\par 由 \eqref{6}式,两式分式对等:
\begin{align*}
    A & =\left(A\setminus A_{1}\right) \cup \left(A_{1}\setminus A_{2}\right) \cup \left(A_{2}\setminus A_{3}\right) \cup \dots \cup D\\
    A_{1} & =\left(A_{1}\setminus A_{2}\right) \cup \left(A_{2}\setminus A_{3}\right) \cup \left(A_{3}\setminus A_{4}\right) \cup \dots \cup D_{1}
\end{align*}
\par 由引理知:
$$A \thicksim A_{1}$$
\par 又因$B\thicksim A_{1}$,由Th \ 1知:
$$A\thicksim B$$

\begin{la}
    \pmb{推论:}若$A\supset B\supset C$且$A \thicksim C$,则$A\thicksim B\thicksim C$   
\end{la}
\noindent 证明:
\par 已知$A \thicksim C$故:
$$\exists \varphi: A \stackrel{1-1}{\to} C$$
\par 令$C^{*} = \varphi (B) \subset \varphi (A) = C$,则:
$$\varphi: B \stackrel{1-1}{\to} C^{*}$$
\par 故:$B\thicksim C^{*}\subset C$
\par 又因:$C\thicksim C\subset B$
\par 由Th2 \tiny{\pageref{Th2} } \normalsize 知,$B \thicksim C$